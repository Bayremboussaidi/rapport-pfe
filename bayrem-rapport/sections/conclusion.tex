\chapter*{Conclusion générale}
\addcontentsline{toc}{chapter}{Conclusion générale}
\markboth{\textbf{Conclusion générale}}{}

Ce projet avait pour objectif de concevoir, développer et déployer une plateforme web complète de location de voitures, adaptée aux besoins de différents utilisateurs : administrateurs, agences, clients et visiteurs. En suivant une méthodologie Agile Scrum et en s’appuyant sur des outils DevOps modernes, nous avons réussi à mener à bien toutes les phases, depuis l’analyse des besoins jusqu’au déploiement sur un cluster Kubernetes.

La plateforme intègre avec succès plusieurs fonctionnalités clés :
\begin{itemize}
    \item Authentification sécurisée multi-rôles et contrôle d’accès via Keycloak.
    \item Gestion des véhicules par les administrateurs et agences, incluant la gestion des photos.
    \item Processus de réservation fluide avec vérification en temps réel de la disponibilité et validation par l’agence.
    \item Notifications automatiques par email et génération de contrats PDF avec QR code intégré.
    \item Système de chat en temps réel facilitant la communication entre agences et administrateurs.
    \item Chatbot intelligent offrant un support 24h/24 via une intégration FastAPI et OpenAI.
    \item Pipelines CI/CD robustes utilisant GitLab, la containerisation Docker et un déploiement scalable via Kubernetes.
\end{itemize}

Ce projet a permis de combiner avec succès architecture RESTful, modularité microservices et outils d’intelligence artificielle pour offrir une solution complète, scalable et maintenable. Il ouvre la voie à de nombreuses améliorations futures tant techniques que fonctionnelles.

\bigskip

Ainsi, ce travail illustre parfaitement comment les technologies web modernes et les principes du cloud peuvent transformer un processus traditionnel de location en une expérience numérique avancée, apportant valeur ajoutée aux utilisateurs et aux gestionnaires du service.

