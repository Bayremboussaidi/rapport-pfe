\chapter*{General Introduction}
\addcontentsline{toc}{chapter}{Introduction générale} % to include the introduction to the table of content
\markboth{Introduction générale}{} % To redefine the section page head

The automotive rental industry has undergone significant transformation in recent years, driven by the increasing demand for digital solutions that streamline operations and enhance user experience. In this context, we developed \textbf{MyLoc}, a comprehensive car rental platform designed to connect administrators, rental agencies, and customers in a seamless digital ecosystem.

Our platform provides a complete solution for car rental management, enabling agencies to list their vehicles, manage their inventory, and process rental requests efficiently. Administrators oversee the entire platform, managing agencies, customers, and content while maintaining direct communication with agencies through an integrated real-time chat system.

The rental workflow is streamlined: when a customer submits a rental request for a specific vehicle, the responsible agency receives the request and can either approve or decline it. Upon approval, the customer receives an email containing a PDF rental contract with all booking details and a unique QR code for verification. In case of rejection, a notification email is sent to inform the customer of the decision. Additionally, in-app notifications keep customers informed of their request status in real-time.

The platform also features a blog section where administrators can publish news and special offers, with customers able to comment and engage with the content. A \textbf{Follower subscription system} allows visitors to enter their email in the website footer to receive automatic email notifications whenever new cars are added to the platform.

To enhance customer support, we integrated an AI-powered chatbot using \textbf{Flask API} (Python) connected to ChatGPT, providing instant answers to visitor and customer inquiries in multiple languages.

Our application is built with modern technologies: \textbf{Angular} for the frontend, \textbf{Spring Boot} (REST architecture) for the backend, and \textbf{MySQL} for data persistence. Security is managed through \textbf{Keycloak} for authentication, with AuthGuard protecting admin and agency routes based on user roles.

For deployment, the entire infrastructure is containerized using \textbf{Docker} and \textbf{Docker Compose}, then orchestrated on \textbf{Kubernetes} (Minikube) with dedicated deployment files for each service. A complete \textbf{CI/CD pipeline} using \textbf{GitLab CI} automates the build, test, and deployment processes.

Development was conducted using \textbf{Visual Studio Code} as the IDE, \textbf{XAMPP} for local database management, and \textbf{Postman} for API testing. The project followed the \textbf{Scrum} methodology with organized sprints and backlogs to ensure iterative and reliable delivery.

This report details the design, development, and deployment approaches adopted for the MyLoc platform. It is organized as follows:

\begin{itemize}
    \item \textbf{Chapter 1}: Presents the project framework and general context, analyzes existing car rental solutions, and introduces the Scrum methodology adopted for project management.
    \item \textbf{Chapter 2}: Covers requirements analysis and specification, including actor identification, functional and non-functional requirements, use case diagrams, class diagrams, and system architecture.
    \item \textbf{Chapter 3}: Details the first release implementation focusing on user authentication (Keycloak integration), car management, and agency management features.
    \item \textbf{Chapter 4}: Presents the second release with rental request processing, email notifications, PDF contract generation with QR codes, and the real-time chat system.
    \item \textbf{Chapter 5}: Covers the third release including the AI chatbot integration, blog management, customer reviews, and notification system.
    \item \textbf{Chapter 6}: Describes the DevOps implementation with Docker containerization, Kubernetes deployment, and GitLab CI/CD pipeline setup.
\end{itemize}
