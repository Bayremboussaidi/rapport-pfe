\chapter{Needs Analysis and Specification}

\section{Introduction}
This chapter formally defines and analyzes the functional and non-functional needs of the car rental platform. By identifying limitations in existing systems, this section translates user expectations into specific system requirements, focusing on secure, real-time, and scalable services for car rentals, including role-based interactions, authentication, booking workflows, communication, and deployment. Following the context and problem analysis outlined in Chapter 1, this chapter ensures that the platform meets real user needs efficiently and securely.

The requirements analysis follows a systematic approach:
\begin{enumerate}
    \item \textbf{Actor Identification}: Defining who interacts with the system and their responsibilities.
    \item \textbf{Functional Requirements}: Specifying what the system must do for each actor.
    \item \textbf{Non-Functional Requirements}: Defining quality attributes like performance, security, and scalability.
    \item \textbf{Use Case Modeling}: Visualizing interactions between actors and the system.
    \item \textbf{Class Modeling}: Defining the data structures and relationships.
\end{enumerate}

\section{Actors Identification}
An actor represents an external entity that interacts with the system. Actors can be humans (users with different roles) or systems (automated services). Understanding actors is crucial for defining system boundaries and requirements.

The MyLoc platform supports the following primary actors:

\subsection{Human Actors}

\subsubsection{Administrator}
The Administrator is the highest-privilege user responsible for the overall management and oversight of the platform.

\textbf{Role Description:}
\begin{itemize}
    \item Manages the entire platform ecosystem including agencies, customers, and content.
    \item Ensures platform security and enforces access control policies.
    \item Monitors system health and performance metrics.
    \item Coordinates with agencies through the built-in chat system.
    \item Manages blog content to engage users and promote services.
\end{itemize}

\textbf{Key Permissions:}
\begin{itemize}
    \item Full CRUD operations on agencies, customers, and blog posts.
    \item Access to all dashboards and administrative panels.
    \item System configuration and settings management.
    \item Direct communication with all agencies.
\end{itemize}

\subsubsection{Agency}
Agencies are the car rental businesses that list their vehicles on the platform and manage customer bookings.

\textbf{Role Description:}
\begin{itemize}
    \item Represents car rental companies that offer vehicles for rent.
    \item Manages their own fleet of cars with full control over listings.
    \item Processes customer rental requests (approve/reject).
    \item Communicates with administrators for support and coordination.
    \item Tracks booking history and manages availability calendars.
\end{itemize}

\textbf{Key Permissions:}
\begin{itemize}
    \item Full CRUD operations on their own cars only.
    \item View and process rental requests for their vehicles.
    \item Chat with administrators.
    \item Manage agency profile and settings.
\end{itemize}

\subsubsection{Customer}
Customers are registered users who browse cars and make rental requests.

\textbf{Role Description:}
\begin{itemize}
    \item Primary consumers of the car rental service.
    \item Can browse the full catalog and make informed decisions.
    \item Submit rental requests with date ranges and preferences.
    \item Track their booking history and request status.
    \item Engage with the platform through comments, reviews, and the chatbot.
\end{itemize}

\textbf{Key Permissions:}
\begin{itemize}
    \item Browse all public content (cars, blogs).
    \item Submit and manage their rental requests.
    \item View their booking history and notifications.
    \item Post comments and reviews.
    \item Contact administrators via email.
    \item Use the AI chatbot for assistance.
\end{itemize}

\subsubsection{Visitor}
Visitors are unauthenticated users exploring the platform before registration.

\textbf{Role Description:}
\begin{itemize}
    \item Potential customers browsing the platform without an account.
    \item Can explore available cars and read blog content.
    \item Can interact with the chatbot for information.
    \item May subscribe as a Follower to receive updates.
    \item Can initiate registration to become a Customer.
\end{itemize}

\textbf{Key Permissions:}
\begin{itemize}
    \item View public car listings (limited details).
    \item Read blog posts.
    \item Use the chatbot.
    \item Subscribe as a Follower.
    \item Register for a full account.
\end{itemize}

\subsubsection{Follower}
Followers are visitors who subscribe to receive email updates without creating a full account.

\textbf{Role Description:}
\begin{itemize}
    \item Interested users who want to stay informed about new offerings.
    \item Subscribe by entering their email in the website footer.
    \item Receive automatic notifications when new cars are added.
    \item Represents a marketing channel for the platform.
    \item Low-commitment engagement method to attract potential customers.
\end{itemize}

\textbf{Key Permissions:}
\begin{itemize}
    \item Subscribe to new car alerts.
    \item Receive email notifications.
    \item Unsubscribe at any time.
\end{itemize}

\subsection{System Actors}

\subsubsection{Chatbot (AI Assistant)}
The Chatbot is an automated AI-powered assistant that provides instant support to users.

\textbf{System Description:}
\begin{itemize}
    \item Built with Flask/FastAPI backend connected to ChatGPT API.
    \item Provides 24/7 automated assistance to visitors and customers.
    \item Answers common questions about cars, bookings, and platform usage.
    \item Supports multiple languages for international users.
    \item Reduces workload on human support staff.
\end{itemize}

\textbf{Capabilities:}
\begin{itemize}
    \item Natural language processing for user queries.
    \item Context-aware responses based on conversation history.
    \item Integration with platform data for accurate information.
    \item Fallback to human support for complex issues.
\end{itemize}

\subsection{Actor Hierarchy Diagram}
Figure~\ref{fig:actor_hierarchy} shows the relationships between different actors and their inheritance of permissions.

\begin{figure}[h!]
    \centering
    \includegraphics[width=10cm]{img/actor-hierarchy.png}
    \caption{Actor Hierarchy and Permission Inheritance}
    \label{fig:actor_hierarchy}
\end{figure}

\section{Needs Identification}

\subsection{Functional Needs}
Functional requirements define what the system must do—the features and capabilities it provides to each actor. These requirements are derived from user stories and stakeholder interviews.

\subsubsection{Customer Functional Requirements}
Customers are the primary end-users of the platform. Their requirements focus on ease of use, transparency, and efficient booking processes.

\begin{itemize}
    \item \textbf{Browsing and Discovery:}
    \begin{itemize}
        \item Browse available cars and blogs without logging in.
        \item Search and filter cars based on availability, type, price, and specifications.
        \item View detailed car information including photos, features, and pricing.
        \item Read reviews and ratings from other customers.
    \end{itemize}

    \item \textbf{Account Management:}
    \begin{itemize}
        \item Register with email verification for security.
        \item Log in securely using Keycloak authentication.
        \item Update profile information and preferences.
        \item Reset password through secure email link.
    \end{itemize}

    \item \textbf{Booking and Rental:}
    \begin{itemize}
        \item Check real-time car availability for selected dates.
        \item Submit rental requests with date range selection.
        \item Receive immediate validation of date availability.
        \item Track rental request status (pending, approved, rejected).
        \item Cancel pending requests before processing.
        \item View complete booking history.
    \end{itemize}

    \item \textbf{Communication and Support:}
    \begin{itemize}
        \item Receive email notifications on rental request status changes.
        \item Receive PDF contract with QR code upon approval.
        \item Contact admin directly via email within the app.
        \item Interact with the AI chatbot for instant support.
        \item Receive in-app notifications for important updates.
    \end{itemize}

    \item \textbf{Engagement:}
    \begin{itemize}
        \item Comment and provide reviews on cars and blog posts.
        \item Receive updates via email when new cars are added.
        \item Share content on social media platforms.
    \end{itemize}
\end{itemize}

\subsubsection{Agency Functional Requirements}
Agencies require tools to efficiently manage their fleet and customer interactions.

\begin{itemize}
    \item \textbf{Account and Profile:}
    \begin{itemize}
        \item Register as an agency with business details.
        \item Log in securely with role-based authentication.
        \item Manage agency profile, logo, and contact information.
    \end{itemize}

    \item \textbf{Fleet Management:}
    \begin{itemize}
        \item Add new cars with complete details (name, model, type, price, features).
        \item Upload multiple photos for each car.
        \item Edit car information and availability.
        \item Delete cars from the platform.
        \item Manage car availability calendars.
    \end{itemize}

    \item \textbf{Booking Management:}
    \begin{itemize}
        \item View all rental requests for their cars.
        \item Accept or decline requests with one click.
        \item Automatic email sent to customers upon decision.
        \item Track booking history and statistics.
    \end{itemize}

    \item \textbf{Communication:}
    \begin{itemize}
        \item Chat with administrators in real-time.
        \item Receive notifications of new rental requests.
        \item View message history for reference.
    \end{itemize}
\end{itemize}

\subsubsection{Administrator Functional Requirements}
Administrators need comprehensive control over all platform aspects.

\begin{itemize}
    \item \textbf{Agency Management:}
    \begin{itemize}
        \item Add new agencies to the platform.
        \item Edit agency information and status.
        \item Delete or suspend agency accounts.
        \item View agency performance metrics.
    \end{itemize}

    \item \textbf{Customer Management:}
    \begin{itemize}
        \item View all registered customers.
        \item Edit customer information if needed.
        \item Delete or suspend customer accounts.
        \item Handle customer complaints and issues.
    \end{itemize}

    \item \textbf{Content Management:}
    \begin{itemize}
        \item Create blog posts with rich content.
        \item Edit and update existing blogs.
        \item Delete inappropriate content.
        \item Moderate comments and reviews.
    \end{itemize}

    \item \textbf{Communication:}
    \begin{itemize}
        \item Chat with agencies for coordination.
        \item Send platform-wide announcements.
        \item Respond to customer contact emails.
    \end{itemize}

    \item \textbf{System Administration:}
    \begin{itemize}
        \item Monitor system health and performance.
        \item Manage security settings via Keycloak.
        \item Configure platform settings and preferences.
        \item Access logs and audit trails.
    \end{itemize}
\end{itemize}

\subsubsection{Follower Functional Requirements}
\begin{itemize}
    \item Subscribe to updates by entering email in the website footer.
    \item Receive automatic email notifications when new cars are added.
    \item Stay informed about new offerings without a full account.
    \item Unsubscribe from notifications at any time via email link.
\end{itemize}

\subsubsection{System-wide Functionalities}
\begin{itemize}
    \item Secure authentication and role-based authorization via Keycloak.
    \item Instant messaging system enabling admin-agency communication.
    \item Email notification system for updates, rental request status, and confirmations.
    \item PDF generation with QR codes for confirmed rental contracts.
    \item Integration of chatbot powered by Flask API for user assistance.
    \item Deployment automation using GitLab CI/CD pipelines, Docker Compose, and Kubernetes.
    \item Real-time availability checking for cars during rental request submission.
\end{itemize}

\subsection{Non-Functional Needs}
\begin{itemize}
    \item \textbf{Performance:}
    \begin{itemize}
        \item The system must respond to user actions (e.g., browsing cars, submitting rental requests) within 2 seconds under normal load.
        \item Rental availability checks must provide real-time feedback during request submission.
        \item Email notifications and PDF generation should be triggered within 2 minutes after any change in request status.
    \end{itemize}

    \item \textbf{Security:}
    \begin{itemize}
        \item Authentication and authorization must be enforced using Keycloak, with role-based access control (Admin, Agency, Customer).
        \item All sensitive data (e.g., passwords, personal information) must be securely stored and transmitted using encryption protocols (HTTPS, password hashing).
        \item All API endpoints must be protected from unauthorized access using AuthGuard and secure access control mechanisms.
        \item The application must defend against common security threats such as SQL injection, cross-site scripting (XSS), and cross-site request forgery (CSRF).
    \end{itemize}

    \item \textbf{Usability:}
    \begin{itemize}
        \item The user interface must be intuitive, user-friendly, and responsive across desktop, tablet, and mobile devices.
        \item Customers must be able to easily browse, search for cars, submit rental requests, and track their status.
        \item Agencies and administrators should have dedicated dashboards for managing cars, requests, blogs, and communication.
    \end{itemize}

    \item \textbf{Reliability and Availability:}
    \begin{itemize}
        \item The system should ensure 99.9\% uptime, with robust error handling and recovery mechanisms.
        \item All rental and user data must be reliably saved and backed up regularly.
    \end{itemize}

    \item \textbf{Maintainability and Extensibility:}
    \begin{itemize}
        \item The codebase must follow clean architecture principles and separation of concerns to facilitate future updates.
        \item Continuous integration and deployment (CI/CD) pipelines using GitLab must support automated testing and deployment.
        \item Comprehensive documentation should be maintained for the API, deployment process, and system usage.
    \end{itemize}

    \item \textbf{Scalability:}
    \begin{itemize}
        \item The system must support increased user traffic and rental activity without performance degradation.
        \item Kubernetes-based deployment must allow for horizontal scaling of services as demand grows.
    \end{itemize}

    \item \textbf{Interoperability:}
    \begin{itemize}
        \item The system must integrate seamlessly with external services, including the Flask-based chatbot API and payment gateways.
        \item Email services must be compatible with SMTP or third-party providers to ensure reliable communication.
    \end{itemize}
\end{itemize}

\section{System Design}

This section presents the visual models that describe the system's structure and behavior. These diagrams follow the Unified Modeling Language (UML) standard and serve as blueprints for development.

\subsection{Use Case Diagrams}
Use case diagrams describe the functional requirements of the system from the user's perspective. They show what the system does, not how it does it. Each diagram represents the interactions between a specific actor and the system.

\subsubsection{Global Use Case Overview}
The global use case diagram represents the overall interactions between primary actors (Customer, Agency, Administrator, Visitor, Follower) and the car rental platform. It highlights the key functionalities accessible to each actor, offering a high-level overview of the system's capabilities.

\textbf{Key System Boundaries:}
\begin{itemize}
    \item \textbf{Public Area}: Accessible to all users (visitors, customers, followers) - includes car browsing, blog reading, and chatbot interaction.
    \item \textbf{Customer Area}: Requires customer authentication - includes booking, history, notifications.
    \item \textbf{Agency Area}: Requires agency authentication - includes fleet management, request processing.
    \item \textbf{Admin Area}: Requires administrator privileges - includes full platform management.
\end{itemize}

\subsubsection{Customer Use Case Diagram}
The customer use case diagram illustrates the primary interactions between customers and the car rental platform. It shows how registered customers can manage their profiles, search and book vehicles, interact with the AI chatbot, and engage with the platform's content system.

\begin{figure}[h!]
    \centering
    \includegraphics[width=16cm]{img/customer-usecase-diagram.png}
    \caption{Use Case Diagram for Customer - Myloc Agency Platform}
    \label{fig:customer_interactions}
\end{figure}

\textbf{Primary Use Cases for Customer:}
\begin{itemize}
    \item \textbf{UC-C01: Browse Cars} - Customer explores the available car catalog with filtering options.
    \item \textbf{UC-C02: View Car Details} - Customer examines detailed information, photos, and pricing.
    \item \textbf{UC-C03: Check Availability} - System validates car availability for selected dates.
    \item \textbf{UC-C04: Submit Rental Request} - Customer submits a booking request for a specific car and date range.
    \item \textbf{UC-C05: Track Request Status} - Customer monitors the status of submitted requests.
    \item \textbf{UC-C06: View Booking History} - Customer accesses past and current bookings.
    \item \textbf{UC-C07: Receive Notifications} - Customer receives email and in-app notifications.
    \item \textbf{UC-C08: Use Chatbot} - Customer interacts with AI assistant for support.
    \item \textbf{UC-C09: Post Comments} - Customer leaves reviews on cars and blogs.
    \item \textbf{UC-C10: Contact Admin} - Customer sends direct email to administrators.
\end{itemize}

\subsubsection{Agency Use Case Diagram}
The agency use case diagram demonstrates how car rental agencies interact with the platform to manage their fleet, handle customer bookings, and maintain their business operations through the comprehensive agency dashboard.

\begin{figure}[h!]
    \centering
    \includegraphics[width=16cm]{img/agency-usecase-diagram.png}
    \caption{Use Case Diagram for Agency - Fleet and Booking Management}
    \label{fig:agency_interactions}
\end{figure}

\textbf{Primary Use Cases for Agency:}
\begin{itemize}
    \item \textbf{UC-A01: Manage Profile} - Agency updates their business information and settings.
    \item \textbf{UC-A02: Add Car} - Agency adds a new vehicle to their fleet with details and photos.
    \item \textbf{UC-A03: Edit Car} - Agency modifies existing car information.
    \item \textbf{UC-A04: Delete Car} - Agency removes a car from the platform.
    \item \textbf{UC-A05: Upload Photos} - Agency uploads multiple images for each car.
    \item \textbf{UC-A06: View Requests} - Agency sees all rental requests for their cars.
    \item \textbf{UC-A07: Approve Request} - Agency accepts a customer's booking request.
    \item \textbf{UC-A08: Reject Request} - Agency declines a customer's booking request.
    \item \textbf{UC-A09: Chat with Admin} - Agency communicates with administrators in real-time.
    \item \textbf{UC-A10: Receive Notifications} - Agency receives alerts for new requests.
\end{itemize}

\subsubsection{Administrator Use Case Diagram}
The administrator use case diagram showcases the comprehensive system management capabilities available to platform administrators, including user management, content moderation, financial oversight, and system maintenance operations.

\begin{figure}[h!]
    \centering
    \includegraphics[width=16cm]{img/admin-usecase-diagram.png}
    \caption{Use Case Diagram for Administrator - System Management}
    \label{fig:admin_interactions}
\end{figure}

\textbf{Primary Use Cases for Administrator:}
\begin{itemize}
    \item \textbf{UC-AD01: Manage Agencies} - Admin performs CRUD operations on agency accounts.
    \item \textbf{UC-AD02: Manage Customers} - Admin manages customer accounts and issues.
    \item \textbf{UC-AD03: Create Blog} - Admin publishes new blog content.
    \item \textbf{UC-AD04: Edit Blog} - Admin updates existing blog posts.
    \item \textbf{UC-AD05: Delete Blog} - Admin removes inappropriate content.
    \item \textbf{UC-AD06: Chat with Agencies} - Admin communicates with agencies in real-time.
    \item \textbf{UC-AD07: View Statistics} - Admin monitors platform metrics and performance.
    \item \textbf{UC-AD08: Manage Settings} - Admin configures system preferences.
    \item \textbf{UC-AD09: Respond to Contacts} - Admin replies to customer inquiries.
    \item \textbf{UC-AD10: Monitor Security} - Admin oversees authentication and access control.
\end{itemize}

\subsubsection{Visitor and Follower Use Cases}
\textbf{Visitor Use Cases:}
\begin{itemize}
    \item \textbf{UC-V01: Browse Public Content} - View car listings and blog posts.
    \item \textbf{UC-V02: Use Chatbot} - Interact with AI assistant.
    \item \textbf{UC-V03: Register Account} - Create a new customer account.
    \item \textbf{UC-V04: Subscribe as Follower} - Enter email for notifications.
\end{itemize}

\textbf{Follower Use Cases:}
\begin{itemize}
    \item \textbf{UC-F01: Subscribe} - Enter email in footer form.
    \item \textbf{UC-F02: Receive Alerts} - Get email when new cars are added.
    \item \textbf{UC-F03: Unsubscribe} - Opt out of notifications via email link.
\end{itemize}

\subsection{Class Diagram}
The class diagram represents the static structure of the system, showing the classes, their attributes, methods, and the relationships between them. It serves as the foundation for the database schema and backend models.

\begin{figure}[h!]
    \centering
    \includegraphics[width=\textwidth]{img/class-diagram.png}
    \caption{Global Class Diagram Representing Car Rental Platform Architecture}
    \label{fig:class_diagram}
\end{figure}

\textbf{Core Entity Classes:}
\begin{itemize}
    \item \textbf{User}: Base class for all user types with common attributes (id, email, password, role).
    \item \textbf{Customer}: Extends User with customer-specific attributes (name, phone, address).
    \item \textbf{Agency}: Extends User with agency details (company name, logo, description).
    \item \textbf{Admin}: Extends User with administrative privileges.
    \item \textbf{Follower}: Simple entity with email for newsletter subscriptions.
    \item \textbf{Car}: Represents vehicles with attributes (name, model, type, price, images, availability).
    \item \textbf{RentalRequest}: Links Customer to Car with dates, status, and timestamps.
    \item \textbf{Blog}: Content entity with title, body, images, and author reference.
    \item \textbf{Comment}: User feedback linked to Car or Blog.
    \item \textbf{Message}: Chat messages between Admin and Agency.
    \item \textbf{Notification}: Alerts for users with type, content, and read status.
\end{itemize}

\textbf{Key Relationships:}
\begin{itemize}
    \item Agency $\rightarrow$ Car: One-to-Many (Agency owns multiple cars).
    \item Customer $\rightarrow$ RentalRequest: One-to-Many (Customer makes multiple requests).
    \item Car $\rightarrow$ RentalRequest: One-to-Many (Car can have multiple requests).
    \item User $\rightarrow$ Comment: One-to-Many (User posts multiple comments).
    \item Admin $\rightarrow$ Blog: One-to-Many (Admin creates multiple blogs).
    \item Admin $\leftrightarrow$ Agency (via Message): Many-to-Many communication.
\end{itemize}

% =============================================================
% PROJECT MANAGEMENT METHODOLOGY
% =============================================================
\section{Project Management Methodology}

\subsection{Traditional vs Agile Methods}
Before selecting a methodology for this project, it is essential to understand the fundamental differences between traditional (classical) and agile approaches to project management. The following table presents a comprehensive comparison:

\begin{table}[h!]
\centering
\small
\renewcommand{\arraystretch}{1.5}
\begin{tabular}{|p{2.8cm}|p{5.2cm}|p{5.2cm}|}
\hline
\rowcolor{gray!20}
\textbf{Criteria} & \textbf{Traditional Methods} & \textbf{Agile Methods} \\
\hline
Process & Linear, sequential (V-Model, Waterfall) & Iterative, incremental (Scrum, Kanban) \\
\hline
Requirements & Detailed specs fixed at project start & Evolving specs, adjusted continuously \\
\hline
Flexibility & Low, hard to integrate changes & High, continuous adaptation \\
\hline
Delivery & Single delivery at project end & Frequent functional increments \\
\hline
Client Interaction & Limited after requirements phase & Continuous collaboration \\
\hline
Risk Management & Upfront identification, few adjustments & Continuous with iterative adjustments \\
\hline
Planning & Complete upfront, rigid & Flexible, adjusted per iteration \\
\hline
Ideal Project & Simple, well-defined, stable & Complex, evolving requirements \\
\hline
Examples & V-Model, Waterfall, PERT & Scrum, Kanban, XP \\
\hline
\end{tabular}
\caption{Comparison between Traditional and Agile Methods}
\label{tab:traditional_vs_agile}
\end{table}

\subsection{Methodology Selection}
The choice of the Scrum methodology for our project is based on several fundamental considerations. Before detailing the advantages of Scrum, it is relevant to compare this methodology with other recognized agile approaches, such as \textbf{Kanban} and \textbf{Extreme Programming (XP)}.

The following comparative table presents the three agile methods and justifies the choice of Scrum:

\begin{table}[h!]
\centering
\small
\renewcommand{\arraystretch}{1.5}
\begin{tabular}{|p{2.2cm}|p{3.6cm}|p{3.6cm}|p{3.6cm}|}
\hline
\rowcolor{gray!20}
\textbf{Criteria} & \textbf{Scrum} & \textbf{Kanban} & \textbf{XP} \\
\hline
Structure & Fixed sprints (10-15 days), defined roles \& ceremonies & Visual board, no fixed iterations & Short iterations, rigorous practices \\
\hline
Flexibility & Adaptation at sprint end & Real-time priority changes & Flexible but requires rigor \\
\hline
Collaboration & Strong via ceremonies \& retrospectives & Continuous but less structured & Intense (pair programming) \\
\hline
Tracking & Burndown charts, Daily Scrum & Kanban board & Frequent feedback \\
\hline
Ideal For & Frequent deliveries & Continuous flow projects & High-quality software \\
\hline
Adaptability & High at sprint boundaries & Continuous & Feedback-driven \\
\hline
\end{tabular}
\caption{Comparison between Agile Methods: Scrum, Kanban, and XP}
\label{tab:agile_comparison}
\end{table}

\textbf{Justification for Choosing Scrum:}
\begin{itemize}
    \item \textbf{Scrum} is chosen for its clear structure, which combines flexibility and rigor, while promoting close collaboration and continuous adaptation at each sprint.
    \item This methodology is particularly suited to our project, which requires regular deliveries and strong interaction with stakeholders.
    \item \textbf{Kanban} is flexible but lacks structure for projects requiring well-defined iterations.
    \item \textbf{XP} is very effective for code quality but demands rigorous discipline and is more oriented towards pure software development.
\end{itemize}

\subsection{Scrum Methodology Overview}
Scrum is an agile project management method designed to improve team productivity, even remotely, while allowing continuous product optimization through regular user feedback. Inspired by rugby, where teams gather in a scrum, Scrum encourages a dynamic and participative approach to project management.

This ensures a fair balance between initial investment and the final delivered product, offering flexibility to redirect the project along the way. Scrum is widely adopted by development teams because it promotes:
\begin{itemize}
    \item Collaboration with the client
    \item Acceptance of change
    \item Interaction between team members
    \item Delivery of operational software
\end{itemize}

\begin{figure}[h!]
    \centering
    \includegraphics[width=0.9\textwidth]{img/scrum.PNG}
    \caption{Scrum Methodology Overview}
    \label{fig:scrum_methodology}
\end{figure}

\subsubsection{Scrum Key Concepts}

\textbf{Sprint}\\
Scrum breaks down the project into different iterations called sprints. Each sprint lasts no more than four weeks, during which the team develops the features specified in the product backlog.

\vspace{0.5cm}
\textbf{Daily Scrum (Daily Stand-up)}\\
This is a 15-minute daily meeting that tracks project progress. During this meeting, team members present:
\begin{itemize}
    \item Tasks completed yesterday
    \item Tasks planned for today
    \item Identified obstacles preventing them from reaching their goal
\end{itemize}

\vspace{0.5cm}
\textbf{Sprint Planning Meeting}\\
A sprint planning meeting lasting no more than 4 hours, during which the team selects the priority features to develop for the upcoming sprint.

\vspace{0.5cm}
\textbf{Sprint Review}\\
The sprint review is a meeting at the end of the sprint where the team presents the developed features to the client and collects corresponding feedback.

\vspace{0.5cm}
\textbf{Sprint Retrospective}\\
A meeting where the team reflects on the sprint to identify what went well, what could be improved, and action items for the next sprint.

\subsubsection{Scrum Team Composition}
The Scrum team consists of the Scrum Master, the Product Owner, and the Development Team. The team's objective is to have business value to display at the end of each sprint and to make the most profitable increments at each sprint.

\begin{table}[h!]
\centering
\begin{tabular}{|p{4cm}|p{10cm}|}
\hline
\textbf{Role} & \textbf{Description} \\
\hline
\textbf{Product Owner (PO)} & Guardian of the project and representative of the client and their needs. Responsible for maximizing the value of the product and managing the product backlog. \\
\hline
\textbf{Scrum Master (SM)} & Facilitator of project realization, team animator focused on the Scrum framework to ensure its proper implementation. Removes impediments and coaches the team. \\
\hline
\textbf{Development Team} & Handles multidisciplinary realizations, is self-organized, and is responsible for the delivered increment. Cross-functional team capable of delivering potentially shippable product increments. \\
\hline
\end{tabular}
\caption{Scrum Team Roles and Responsibilities}
\label{tab:scrum_roles}
\end{table}

The Scrum team for this project is presented in the following table:

\begin{table}[h!]
\centering
\begin{tabular}{|c|c|c|}
\hline
\textbf{Product Owner} & \textbf{Scrum Master} & \textbf{Development Team} \\
\hline
Amine Ben Chaabane & Mme. Sawsen Jalel & Bayrem Boussaidi \\
\hline
\end{tabular}
\caption{Project Scrum Team}
\label{tab:project_scrum_team}
\end{table}

% =============================================================
% SCRUM IMPLEMENTATION DETAILS
% =============================================================
\section{Scrum Implementation and Artifacts}
This section presents the detailed Scrum artifacts used throughout the project, including the product backlog, sprint planning, burndown charts, and velocity tracking—similar to tools like Jira.

\begin{figure}[h!]
    \centering
    \includegraphics[width=12cm]{img/scrum-team-structure.png}
    \caption{Scrum Team Structure - Myloc Agency Project}
    \label{fig:scrum_team}
\end{figure}

\begin{figure}[h!]
    \centering
    \includegraphics[width=14cm]{img/scrum-process-flow.png}
    \caption{Scrum Process Flow}
    \label{fig:scrum_process}
\end{figure}

% Include the comprehensive Scrum artifacts
% =============================================================
% SCRUM ARTIFACTS - Car Rental Platform (Myloc Agency)
% =============================================================
% This file contains all Scrum documentation including:
% - Product Backlog
% - Sprint Planning
% - Sprint Backlogs
% - Burndown Chart Data
% - Velocity Tracking
% =============================================================

% -------------------------------------------------------------
% INTRODUCTION TO SCRUM ARTIFACTS
% -------------------------------------------------------------
\section{Introduction to Scrum Artifacts}

Scrum artifacts provide key information that the Scrum Team and stakeholders need to understand the work being done, the value being created, and the progress being made. These artifacts are designed to maximize transparency and provide opportunities for inspection and adaptation.

In the context of the MyLoc car rental platform, Scrum artifacts played a crucial role in organizing development work, tracking progress, and ensuring all stakeholders had visibility into the project status.

\subsection{Purpose of Scrum Artifacts}

\textbf{Product Backlog}: Serves as the single source of truth for all work to be done on the product. It contains all features, requirements, enhancements, and bug fixes needed for the MyLoc platform.

\textbf{Sprint Backlog}: Provides a real-time picture of the work planned for the current sprint, helping the Development Team stay focused on the Sprint Goal.

\textbf{Increment}: Represents the cumulative value delivered at the end of each sprint—a potentially releasable version of the product.

\textbf{Burndown Charts}: Visual tools that helped our team track progress, identify potential delays early, and make data-driven decisions.

\textbf{Velocity Charts}: Helped forecast future capacity and establish realistic sprint commitments based on historical performance.

\subsection{How We Used Artifacts}
Throughout the MyLoc project, these artifacts were:
\begin{itemize}
    \item \textbf{Maintained Continuously}: The Product Backlog was refined weekly, with new items added and priorities adjusted based on stakeholder feedback.
    \item \textbf{Reviewed During Ceremonies}: Sprint Backlogs were created during Sprint Planning and reviewed daily during Stand-ups.
    \item \textbf{Shared Transparently}: All artifacts were accessible to stakeholders, promoting trust and collaboration.
    \item \textbf{Used for Decision Making}: Burndown and velocity data informed sprint planning and release forecasting.
\end{itemize}

% -------------------------------------------------------------
% SECTION: Product Backlog
% -------------------------------------------------------------
\section{Product Backlog}
The Product Backlog is the master list of all features, requirements, enhancements, and fixes that constitute the changes to be made to the product. Each item is described as a User Story with acceptance criteria, priority, and story point estimation.

\subsection{User Stories and Prioritization}
User stories follow the format: \textit{"As a [role], I want [feature] so that [benefit]"}. Priority levels are defined using MoSCoW method (Must have, Should have, Could have, Won't have).

% Table split into two parts for better readability
\subsubsection{Authentication, Car Management \& Booking (Sprints 1-3)}

\begin{table}[h!]
\centering
\small
\renewcommand{\arraystretch}{1.3}
\begin{tabular}{|c|l|c|c|}
\hline
\rowcolor{gray!20}
\textbf{ID} & \textbf{User Story Description} & \textbf{Priority} & \textbf{Sprint} \\
\hline
US-01 & Visitor registers an account & Must & 1 \\
\hline
US-02 & User logs in securely & Must & 1 \\
\hline
US-03 & Admin manages user roles & Must & 1 \\
\hline
US-04 & User resets password & Should & 1 \\
\hline
US-05 & User updates profile & Should & 2 \\
\hline
US-06 & Agency adds new cars & Must & 2 \\
\hline
US-07 & Agency edits car details & Must & 2 \\
\hline
US-08 & Agency deletes cars & Must & 2 \\
\hline
US-09 & Agency uploads car photos & Must & 2 \\
\hline
US-10 & Customer browses cars & Must & 2 \\
\hline
US-11 & Customer filters cars by criteria & Should & 2 \\
\hline
US-12 & Customer views car details & Must & 2 \\
\hline
US-13 & Customer checks car availability & Must & 3 \\
\hline
US-14 & Customer submits rental request & Must & 3 \\
\hline
US-15 & Agency views rental requests & Must & 3 \\
\hline
US-16 & Agency approves/rejects requests & Must & 3 \\
\hline
US-17 & Customer receives email notifications & Must & 3 \\
\hline
\end{tabular}
\caption{Product Backlog - Part 1: Core Features (Sprints 1-3)}
\label{tab:backlog_part1}
\end{table}

\subsubsection{Communication, Blog, Contracts \& DevOps (Sprints 4-8)}

\begin{table}[h!]
\centering
\small
\renewcommand{\arraystretch}{1.3}
\begin{tabular}{|c|l|c|c|}
\hline
\rowcolor{gray!20}
\textbf{ID} & \textbf{User Story Description} & \textbf{Priority} & \textbf{Sprint} \\
\hline
US-18 & Customer views booking history & Should & 4 \\
\hline
US-19 & Customer cancels pending requests & Should & 4 \\
\hline
US-20 & Admin chats with agencies & Must & 4 \\
\hline
US-21 & Agency receives real-time messages & Must & 4 \\
\hline
US-22 & User receives in-app notifications & Should & 4 \\
\hline
US-23 & Customer contacts admin via email & Should & 4 \\
\hline
US-24 & Admin creates blog posts & Should & 5 \\
\hline
US-25 & Admin edits/deletes blogs & Should & 5 \\
\hline
US-26 & Visitor reads blogs & Should & 5 \\
\hline
US-27 & Customer comments on blogs & Could & 5 \\
\hline
US-28 & Visitor becomes Follower (email subscription for new car alerts) & Could & 5 \\
\hline
US-29 & Customer receives PDF contract & Must & 6 \\
\hline
US-30 & Contract includes QR code & Should & 6 \\
\hline
US-31 & Customer pays online & Must & 6 \\
\hline
US-32 & Agency tracks payments & Should & 6 \\
\hline
US-33 & Visitor uses chatbot & Should & 7 \\
\hline
US-34 & Customer uses multilingual chatbot & Could & 7 \\
\hline
US-35 & Admin manages agencies & Must & 7 \\
\hline
US-36 & Admin views statistics & Should & 7 \\
\hline
US-37 & Admin manages customers & Should & 7 \\
\hline
US-38 & Developer creates Docker containers & Must & 8 \\
\hline
US-39 & Developer sets up CI/CD pipelines & Must & 8 \\
\hline
US-40 & Developer deploys on Kubernetes & Should & 8 \\
\hline
\end{tabular}
\caption{Product Backlog - Part 2: Advanced Features (Sprints 4-8)}
\label{tab:backlog_part2}
\end{table}

\subsection{Backlog Summary}
\begin{table}[h!]
\centering
\begin{tabular}{|l|c|c|}
\hline
\textbf{Priority Level} & \textbf{Number of Stories} & \textbf{Total Story Points} \\
\hline
Must Have & 22 & 132 \\
\hline
Should Have & 14 & 63 \\
\hline
Could Have & 4 & 14 \\
\hline
\textbf{Total} & \textbf{40} & \textbf{209} \\
\hline
\end{tabular}
\caption{Product Backlog Summary by Priority}
\end{table}

% -------------------------------------------------------------
% SECTION: Sprint Planning Details
% -------------------------------------------------------------
\section{Sprint Planning}
The project was executed over 8 sprints, each lasting 10-15 days. The team velocity stabilized at approximately 26 story points per sprint after the initial sprints.

\subsection{Sprint Overview}
\begin{table}[h!]
\centering
\renewcommand{\arraystretch}{1.3}
\begin{tabular}{|c|l|c|c|c|}
\hline
\rowcolor{gray!20}
\textbf{Sprint} & \textbf{Goal} & \textbf{Duration} & \textbf{Planned SP} & \textbf{Completed SP} \\
\hline
1 & Authentication \& Setup & 12 days & 24 & 24 \\
\hline
2 & Car Management Module & 14 days & 28 & 28 \\
\hline
3 & Booking System Core & 15 days & 31 & 31 \\
\hline
4 & Communication Features & 12 days & 24 & 24 \\
\hline
5 & Blog \& Subscription & 10 days & 17 & 17 \\
\hline
6 & Contracts \& Payments & 15 days & 31 & 31 \\
\hline
7 & Chatbot \& Admin Panel & 12 days & 21 & 21 \\
\hline
8 & DevOps \& Deployment & 15 days & 34 & 34 \\
\hline
\rowcolor{gray!10}
\multicolumn{3}{|r|}{\textbf{Total (105 days)}} & \textbf{210} & \textbf{210} \\
\hline
\end{tabular}
\caption{Sprint Overview - Planned vs Completed Story Points}
\end{table}

% -------------------------------------------------------------
% SPRINT 1: Authentication & Setup
% -------------------------------------------------------------
\subsection{Sprint 1: Authentication \& Platform Setup}
\textbf{Sprint Goal:} Set up development environment and implement secure multi-role authentication using Keycloak.

\textbf{Duration:} 12 days

\begin{table}[h!]
\centering
\renewcommand{\arraystretch}{1.2}
\begin{tabular}{|c|p{5.5cm}|c|c|c|}
\hline
\rowcolor{gray!20}
\textbf{Task} & \textbf{Description} & \textbf{Assignee} & \textbf{Hours} & \textbf{Status} \\
\hline
T1.1 & Initialize Angular 16 project & Dev & 4h & Done \\
\hline
T1.2 & Set up Spring Boot 3.x backend & Dev & 4h & Done \\
\hline
T1.3 & Configure MySQL database & Dev & 6h & Done \\
\hline
T1.4 & Deploy Keycloak container & Dev & 4h & Done \\
\hline
T1.5 & Implement registration API & Dev & 8h & Done \\
\hline
T1.6 & Implement JWT login & Dev & 10h & Done \\
\hline
T1.7 & Create AuthGuard & Dev & 8h & Done \\
\hline
T1.8 & Password reset feature & Dev & 6h & Done \\
\hline
T1.9 & Login/register UI & Dev & 8h & Done \\
\hline
T1.10 & Unit testing & Dev & 6h & Done \\
\hline
\end{tabular}
\caption{Sprint 1 Backlog - Authentication \& Setup}
\end{table}

\textbf{Sprint 1 Retrospective:}
\begin{itemize}
    \item \textit{What went well:} Keycloak integration was smoother than expected.
    \item \textit{What could improve:} Initial environment setup took longer due to Docker configuration on Windows.
    \item \textit{Action items:} Document Docker setup steps for future reference.
\end{itemize}

% -------------------------------------------------------------
% SPRINT 2: Car Management
% -------------------------------------------------------------
\subsection{Sprint 2: Car Management Module}
\textbf{Sprint Goal:} Implement complete CRUD operations for car listings with image upload functionality.

\textbf{Duration:} 14 days

\begin{table}[h!]
\centering
\renewcommand{\arraystretch}{1.2}
\begin{tabular}{|c|p{5.5cm}|c|c|c|}
\hline
\rowcolor{gray!20}
\textbf{Task} & \textbf{Description} & \textbf{Assignee} & \textbf{Hours} & \textbf{Status} \\
\hline
T2.1 & Design Voiture entity & Dev & 4h & Done \\
\hline
T2.2 & Create car CRUD endpoints & Dev & 8h & Done \\
\hline
T2.3 & Image upload service & Dev & 6h & Done \\
\hline
T2.4 & Car listing component & Dev & 6h & Done \\
\hline
T2.5 & Car detail page & Dev & 4h & Done \\
\hline
T2.6 & Search \& filter functionality & Dev & 8h & Done \\
\hline
T2.7 & Agency dashboard & Dev & 10h & Done \\
\hline
T2.8 & Profile update feature & Dev & 6h & Done \\
\hline
T2.9 & Form validation & Dev & 4h & Done \\
\hline
T2.10 & Integration testing & Dev & 6h & Done \\
\hline
\end{tabular}
\caption{Sprint 2 Backlog - Car Management}
\end{table}

% -------------------------------------------------------------
% SPRINT 3: Booking System
% -------------------------------------------------------------
\subsection{Sprint 3: Booking System Core}
\textbf{Sprint Goal:} Implement the complete rental request workflow with availability checking and email notifications.

\textbf{Duration:} 15 days

\begin{table}[h!]
\centering
\renewcommand{\arraystretch}{1.2}
\begin{tabular}{|c|p{5.5cm}|c|c|c|}
\hline
\rowcolor{gray!20}
\textbf{Task} & \textbf{Description} & \textbf{Assignee} & \textbf{Hours} & \textbf{Status} \\
\hline
T3.1 & Design Booking entity & Dev & 4h & Done \\
\hline
T3.2 & Availability checking algorithm & Dev & 10h & Done \\
\hline
T3.3 & Booking request API & Dev & 8h & Done \\
\hline
T3.4 & Booking form with date picker & Dev & 6h & Done \\
\hline
T3.5 & Agency request management & Dev & 8h & Done \\
\hline
T3.6 & Approve/reject workflow & Dev & 6h & Done \\
\hline
T3.7 & Configure SMTP email & Dev & 4h & Done \\
\hline
T3.8 & Email notification templates & Dev & 6h & Done \\
\hline
T3.9 & Booking history page & Dev & 6h & Done \\
\hline
T3.10 & End-to-end testing & Dev & 8h & Done \\
\hline
\end{tabular}
\caption{Sprint 3 Backlog - Booking System}
\end{table}

% -------------------------------------------------------------
% SPRINT 4: Communication Features
% -------------------------------------------------------------
\subsection{Sprint 4: Communication Features}
\textbf{Sprint Goal:} Implement real-time chat system and notification infrastructure.

\textbf{Duration:} 12 days

\begin{table}[h!]
\centering
\renewcommand{\arraystretch}{1.2}
\begin{tabular}{|c|p{5.5cm}|c|c|c|}
\hline
\rowcolor{gray!20}
\textbf{Task} & \textbf{Description} & \textbf{Assignee} & \textbf{Hours} & \textbf{Status} \\
\hline
T4.1 & Design ChatMessage entity & Dev & 3h & Done \\
\hline
T4.2 & HTTP Polling configuration & Dev & 8h & Done \\
\hline
T4.3 & Chat service \& endpoints & Dev & 6h & Done \\
\hline
T4.4 & Chat UI component & Dev & 8h & Done \\
\hline
T4.5 & Notification entity \& service & Dev & 6h & Done \\
\hline
T4.6 & Notification dropdown & Dev & 4h & Done \\
\hline
T4.7 & Contact admin feature & Dev & 4h & Done \\
\hline
T4.8 & Booking cancel functionality & Dev & 4h & Done \\
\hline
T4.9 & Polish chat interface & Dev & 4h & Done \\
\hline
T4.10 & Real-time testing & Dev & 6h & Done \\
\hline
\end{tabular}
\caption{Sprint 4 Backlog - Communication Features}
\end{table}

% -------------------------------------------------------------
% SPRINT 5: Blog System
% -------------------------------------------------------------
\subsection{Sprint 5: Blog \& Subscription System}
\textbf{Sprint Goal:} Implement blog management and email subscription for visitors.

\textbf{Duration:} 10 days

\begin{table}[h!]
\centering
\renewcommand{\arraystretch}{1.2}
\begin{tabular}{|c|p{5.5cm}|c|c|c|}
\hline
\rowcolor{gray!20}
\textbf{Task} & \textbf{Description} & \textbf{Assignee} & \textbf{Hours} & \textbf{Status} \\
\hline
T5.1 & Design Blog entity & Dev & 3h & Done \\
\hline
T5.2 & Blog CRUD API endpoints & Dev & 6h & Done \\
\hline
T5.3 & Admin blog management & Dev & 6h & Done \\
\hline
T5.4 & Public blog listing & Dev & 4h & Done \\
\hline
T5.5 & Blog detail with comments & Dev & 6h & Done \\
\hline
T5.6 & Comment functionality & Dev & 4h & Done \\
\hline
T5.7 & Follower entity & Dev & 2h & Done \\
\hline
T5.8 & Subscription API & Dev & 4h & Done \\
\hline
T5.9 & Subscription form & Dev & 3h & Done \\
\hline
T5.10 & Feature testing & Dev & 4h & Done \\
\hline
\end{tabular}
\caption{Sprint 5 Backlog - Blog \& Subscription}
\end{table}

% -------------------------------------------------------------
% SPRINT 6: Contracts & Payments
% -------------------------------------------------------------
\subsection{Sprint 6: Contracts \& Payments}
\textbf{Sprint Goal:} Implement PDF contract generation with QR codes and online payment integration.

\textbf{Duration:} 15 days

\begin{table}[h!]
\centering
\renewcommand{\arraystretch}{1.2}
\begin{tabular}{|c|p{5.5cm}|c|c|c|}
\hline
\rowcolor{gray!20}
\textbf{Task} & \textbf{Description} & \textbf{Assignee} & \textbf{Hours} & \textbf{Status} \\
\hline
T6.1 & Research PDF libraries (iText) & Dev & 4h & Done \\
\hline
T6.2 & Contract PDF template & Dev & 6h & Done \\
\hline
T6.3 & PDF generation service & Dev & 10h & Done \\
\hline
T6.4 & QR code integration (ZXing) & Dev & 6h & Done \\
\hline
T6.5 & Payment microservice & Dev & 6h & Done \\
\hline
T6.6 & Payment API endpoints & Dev & 10h & Done \\
\hline
T6.7 & Payment email template & Dev & 4h & Done \\
\hline
T6.8 & Payment webhook & Dev & 6h & Done \\
\hline
T6.9 & Auto-send PDF on payment & Dev & 4h & Done \\
\hline
T6.10 & Payment-to-contract testing & Dev & 8h & Done \\
\hline
\end{tabular}
\caption{Sprint 6 Backlog - Contracts \& Payments}
\end{table}

% -------------------------------------------------------------
% SPRINT 7: Chatbot & Admin
% -------------------------------------------------------------
\subsection{Sprint 7: Chatbot \& Admin Panel}
\textbf{Sprint Goal:} Integrate AI chatbot and complete admin dashboard functionalities.

\textbf{Duration:} 12 days

\begin{table}[h!]
\centering
\renewcommand{\arraystretch}{1.2}
\begin{tabular}{|c|p{5.5cm}|c|c|c|}
\hline
\rowcolor{gray!20}
\textbf{Task} & \textbf{Description} & \textbf{Assignee} & \textbf{Hours} & \textbf{Status} \\
\hline
T7.1 & Flask chatbot project setup & Dev & 4h & Done \\
\hline
T7.2 & OpenAI ChatGPT integration & Dev & 6h & Done \\
\hline
T7.3 & Chatbot REST endpoints & Dev & 4h & Done \\
\hline
T7.4 & Chatbot UI widget & Dev & 6h & Done \\
\hline
T7.5 & Multilingual support (FR/EN) & Dev & 4h & Done \\
\hline
T7.6 & Agency management (Admin) & Dev & 6h & Done \\
\hline
T7.7 & Customer management (Admin) & Dev & 4h & Done \\
\hline
T7.8 & Statistics dashboard & Dev & 6h & Done \\
\hline
T7.9 & Data visualization charts & Dev & 4h & Done \\
\hline
T7.10 & Integration testing & Dev & 4h & Done \\
\hline
\end{tabular}
\caption{Sprint 7 Backlog - Chatbot \& Admin Panel}
\end{table}

% -------------------------------------------------------------
% SPRINT 8: DevOps & Deployment
% -------------------------------------------------------------
\subsection{Sprint 8: DevOps \& Deployment}
\textbf{Sprint Goal:} Containerize all services and establish CI/CD pipeline with Kubernetes deployment.

\textbf{Duration:} 15 days

\begin{table}[h!]
\centering
\renewcommand{\arraystretch}{1.2}
\begin{tabular}{|c|p{5.5cm}|c|c|c|}
\hline
\rowcolor{gray!20}
\textbf{Task} & \textbf{Description} & \textbf{Assignee} & \textbf{Hours} & \textbf{Status} \\
\hline
T8.1 & Dockerfile for Angular & Dev & 4h & Done \\
\hline
T8.2 & Dockerfile for Spring Boot & Dev & 4h & Done \\
\hline
T8.3 & Dockerfile for Flask Chatbot & Dev & 3h & Done \\
\hline
T8.4 & docker-compose.yml & Dev & 6h & Done \\
\hline
T8.5 & GitLab CI/CD pipeline & Dev & 10h & Done \\
\hline
T8.6 & GitLab Container Registry & Dev & 4h & Done \\
\hline
T8.7 & K8s deployment YAMLs & Dev & 10h & Done \\
\hline
T8.8 & K8s services \& ingress & Dev & 6h & Done \\
\hline
T8.9 & Minikube deployment & Dev & 8h & Done \\
\hline
T8.10 & Documentation & Dev & 4h & Done \\
\hline
\end{tabular}
\caption{Sprint 8 Backlog - DevOps \& Deployment}
\end{table}

% -------------------------------------------------------------
% SECTION: Burndown Chart Data
% -------------------------------------------------------------
\section{Sprint Burndown Analysis}
The burndown chart tracks the remaining work (in story points) over the course of each sprint. Figure \ref{fig:burndown_chart} illustrates the project-level burndown across all sprints.

\begin{figure}[h!]
    \centering
    \includegraphics[width=14cm]{img/project-burndown-chart.png}
    \caption{Project Burndown Chart - Story Points Remaining Over Sprints}
    \label{fig:burndown_chart}
\end{figure}

\subsection{Project Burndown Data}
\begin{table}[h!]
\centering
\renewcommand{\arraystretch}{1.3}
\begin{tabular}{|c|c|c|c|}
\hline
\rowcolor{gray!20}
\textbf{Sprint} & \textbf{Start SP} & \textbf{Completed SP} & \textbf{Remaining} \\
\hline
Project Start & 210 & -- & 210 \\
\hline
Sprint 1 (12 days) & 210 & 24 & 186 \\
\hline
Sprint 2 (14 days) & 186 & 28 & 158 \\
\hline
Sprint 3 (15 days) & 158 & 31 & 127 \\
\hline
Sprint 4 (12 days) & 127 & 24 & 103 \\
\hline
Sprint 5 (10 days) & 103 & 17 & 86 \\
\hline
Sprint 6 (15 days) & 86 & 31 & 55 \\
\hline
Sprint 7 (12 days) & 55 & 21 & 34 \\
\hline
Sprint 8 (15 days) & 34 & 34 & 0 \\
\hline
\rowcolor{gray!10}
\textbf{Total (105 days)} & \textbf{210} & \textbf{210} & \textbf{0} \\
\hline
\end{tabular}
\caption{Project Burndown - Story Points per Sprint}
\end{table}

\subsection{Sprint 3 Detailed Burndown (Example)}
The following table shows the daily burndown for Sprint 3 (Booking System), demonstrating how work was completed throughout the 15-day period.

\begin{table}[h!]
\centering
\renewcommand{\arraystretch}{1.2}
\begin{tabular}{|c|c|c|c|}
\hline
\rowcolor{gray!20}
\textbf{Day} & \textbf{Ideal} & \textbf{Actual} & \textbf{Done} \\
\hline
Day 1 & 28.9 & 31 & 0 \\
\hline
Day 2 & 26.8 & 29 & 2 \\
\hline
Day 3 & 24.8 & 26 & 3 \\
\hline
Day 4 & 22.7 & 24 & 2 \\
\hline
Day 5 & 20.7 & 21 & 3 \\
\hline
Day 6 & 18.6 & 18 & 3 \\
\hline
Day 7 & 16.5 & 15 & 3 \\
\hline
Day 8 & 14.5 & 12 & 3 \\
\hline
Day 9 & 12.4 & 9 & 3 \\
\hline
Day 10 & 10.3 & 7 & 2 \\
\hline
Day 11 & 8.3 & 5 & 2 \\
\hline
Day 12 & 6.2 & 3 & 2 \\
\hline
Day 13 & 4.1 & 2 & 1 \\
\hline
Day 14 & 2.1 & 1 & 1 \\
\hline
Day 15 & 0 & 0 & 1 \\
\hline
\end{tabular}
\caption{Sprint 3 Daily Burndown Chart Data}
\end{table}

% -------------------------------------------------------------
% SECTION: Team Velocity
% -------------------------------------------------------------
\section{Team Velocity}
Team velocity measures the amount of work completed per sprint, expressed in story points. This metric helps in future sprint planning and capacity estimation. Figure \ref{fig:velocity_chart} visualizes the velocity trend across all sprints.

\begin{figure}[h!]
    \centering
    \includegraphics[width=14cm]{img/velocity-chart.png}
    \caption{Team Velocity Chart - Story Points Completed per Sprint}
    \label{fig:velocity_chart}
\end{figure}

\begin{table}[h!]
\centering
\renewcommand{\arraystretch}{1.3}
\begin{tabular}{|c|c|c|c|}
\hline
\rowcolor{gray!20}
\textbf{Sprint} & \textbf{Duration} & \textbf{Velocity (SP)} & \textbf{Cumulative SP} \\
\hline
Sprint 1 & 12 days & 24 & 24 \\
\hline
Sprint 2 & 14 days & 28 & 52 \\
\hline
Sprint 3 & 15 days & 31 & 83 \\
\hline
Sprint 4 & 12 days & 24 & 107 \\
\hline
Sprint 5 & 10 days & 17 & 124 \\
\hline
Sprint 6 & 15 days & 31 & 155 \\
\hline
Sprint 7 & 12 days & 21 & 176 \\
\hline
Sprint 8 & 15 days & 34 & 210 \\
\hline
\rowcolor{gray!10}
\multicolumn{2}{|r|}{\textbf{Average Velocity}} & \textbf{26.25 SP} & \textbf{210 SP Total} \\
\hline
\end{tabular}
\caption{Team Velocity per Sprint}
\end{table}

\textbf{Velocity Analysis:}
\begin{itemize}
    \item The team maintained an average velocity of \textbf{26.25 story points per sprint}.
    \item Sprint 5 had lower velocity due to the complexity of blog features being lower than estimated.
    \item Sprint 8 had higher velocity due to familiarity with DevOps tasks and parallel work streams.
    \item Velocity remained consistent, indicating stable team performance and accurate estimation.
\end{itemize}

% -------------------------------------------------------------
% SECTION: Definition of Done
% -------------------------------------------------------------
\section{Definition of Done (DoD)}
A user story is considered "Done" when all the following criteria are met:

\begin{itemize}
    \item Code is written and follows project coding standards.
    \item Unit tests are written and pass successfully.
    \item Code is reviewed by at least one team member.
    \item Feature is integrated with the main branch.
    \item Feature is tested in the staging environment.
    \item Documentation is updated (API docs, README if needed).
    \item No critical or high-severity bugs remain.
    \item Product Owner accepts the feature during Sprint Review.
\end{itemize}

% -------------------------------------------------------------
% SECTION: Scrum Ceremonies Summary
% -------------------------------------------------------------
\section{Scrum Ceremonies}
The following Scrum ceremonies were conducted throughout the project:

\begin{table}[h!]
\centering
\renewcommand{\arraystretch}{1.4}
\begin{tabular}{|p{3.5cm}|p{3cm}|p{7cm}|}
\hline
\rowcolor{gray!20}
\textbf{Ceremony} & \textbf{Frequency} & \textbf{Purpose} \\
\hline
Sprint Planning & Start of sprint & Define sprint goal and select backlog items \\
\hline
Daily Stand-up & Daily (15 min) & Share progress, plans, and blockers \\
\hline
Sprint Review & End of sprint & Demo completed features to stakeholders \\
\hline
Retrospective & End of sprint & Reflect on process and identify improvements \\
\hline
Backlog Refinement & Weekly & Clarify, estimate, and prioritize backlog items \\
\hline
\end{tabular}
\caption{Scrum Ceremonies Overview}
\label{tab:scrum_ceremonies}
\end{table}


\subsection{Chapter Conclusion}
In this chapter, we explored the functional and non-functional requirements for the car rental platform. We identified the different user roles (Admin, Agency, Customer, Visitor), their needs, and outlined the system features accordingly. 

We also established the development methodology using Scrum, assigned team roles (Product Owner, Scrum Master, Development Team), and laid out the complete sprint plan with a detailed product backlog containing 40 user stories totaling 210 story points. The project was organized into 8 sprints with an average team velocity of 26.25 story points per sprint.

This structured specification ensures the platform is developed efficiently, securely, and in alignment with user needs through iterative development and continuous feedback.

\section{Conclusion}
The analysis and specification phase has established a solid foundation for the project. With clearly defined requirements, comprehensive use case diagrams, and a detailed Scrum planning with product backlog, the development team is well-prepared to begin implementation.

In the following chapters, we will detail the realization of each sprint, starting with Sprint 1 focused on platform setup and secure authentication using Keycloak.
