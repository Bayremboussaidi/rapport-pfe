% =============================================================
% SCRUM ARTIFACTS - Car Rental Platform (Myloc Agency)
% =============================================================
% This file contains all Scrum documentation including:
% - Product Backlog
% - Sprint Planning
% - Sprint Backlogs
% - Burndown Chart Data
% - Velocity Tracking
% =============================================================

% -------------------------------------------------------------
% INTRODUCTION TO SCRUM ARTIFACTS
% -------------------------------------------------------------
\section{Introduction to Scrum Artifacts}

Scrum artifacts provide key information that the Scrum Team and stakeholders need to understand the work being done, the value being created, and the progress being made. These artifacts are designed to maximize transparency and provide opportunities for inspection and adaptation.

In the context of the MyLoc car rental platform, Scrum artifacts played a crucial role in organizing development work, tracking progress, and ensuring all stakeholders had visibility into the project status.

\subsection{Purpose of Scrum Artifacts}

\textbf{Product Backlog}: Serves as the single source of truth for all work to be done on the product. It contains all features, requirements, enhancements, and bug fixes needed for the MyLoc platform.

\textbf{Sprint Backlog}: Provides a real-time picture of the work planned for the current sprint, helping the Development Team stay focused on the Sprint Goal.

\textbf{Increment}: Represents the cumulative value delivered at the end of each sprint—a potentially releasable version of the product.

\textbf{Burndown Charts}: Visual tools that helped our team track progress, identify potential delays early, and make data-driven decisions.

\textbf{Velocity Charts}: Helped forecast future capacity and establish realistic sprint commitments based on historical performance.

\subsection{How We Used Artifacts}
Throughout the MyLoc project, these artifacts were:
\begin{itemize}
    \item \textbf{Maintained Continuously}: The Product Backlog was refined weekly, with new items added and priorities adjusted based on stakeholder feedback.
    \item \textbf{Reviewed During Ceremonies}: Sprint Backlogs were created during Sprint Planning and reviewed daily during Stand-ups.
    \item \textbf{Shared Transparently}: All artifacts were accessible to stakeholders, promoting trust and collaboration.
    \item \textbf{Used for Decision Making}: Burndown and velocity data informed sprint planning and release forecasting.
\end{itemize}

% -------------------------------------------------------------
% SECTION: Product Backlog
% -------------------------------------------------------------
\section{Product Backlog}
The Product Backlog is the master list of all features, requirements, enhancements, and fixes that constitute the changes to be made to the product. Each item is described as a User Story with acceptance criteria, priority, and story point estimation.

\subsection{User Stories and Prioritization}
User stories follow the format: \textit{"As a [role], I want [feature] so that [benefit]"}. Priority levels are defined using MoSCoW method (Must have, Should have, Could have, Won't have).

% Table split into two parts for better readability
\subsubsection{Authentication, Car Management \& Booking (Sprints 1-3)}

\begin{table}[h!]
\centering
\small
\renewcommand{\arraystretch}{1.3}
\begin{tabular}{|c|l|c|c|}
\hline
\rowcolor{gray!20}
\textbf{ID} & \textbf{User Story Description} & \textbf{Priority} & \textbf{Sprint} \\
\hline
US-01 & Visitor registers an account & Must & 1 \\
\hline
US-02 & User logs in securely & Must & 1 \\
\hline
US-03 & Admin manages user roles & Must & 1 \\
\hline
US-04 & User resets password & Should & 1 \\
\hline
US-05 & User updates profile & Should & 2 \\
\hline
US-06 & Agency adds new cars & Must & 2 \\
\hline
US-07 & Agency edits car details & Must & 2 \\
\hline
US-08 & Agency deletes cars & Must & 2 \\
\hline
US-09 & Agency uploads car photos & Must & 2 \\
\hline
US-10 & Customer browses cars & Must & 2 \\
\hline
US-11 & Customer filters cars by criteria & Should & 2 \\
\hline
US-12 & Customer views car details & Must & 2 \\
\hline
US-13 & Customer checks car availability & Must & 3 \\
\hline
US-14 & Customer submits rental request & Must & 3 \\
\hline
US-15 & Agency views rental requests & Must & 3 \\
\hline
US-16 & Agency approves/rejects requests & Must & 3 \\
\hline
US-17 & Customer receives email notifications & Must & 3 \\
\hline
\end{tabular}
\caption{Product Backlog - Part 1: Core Features (Sprints 1-3)}
\label{tab:backlog_part1}
\end{table}

\subsubsection{Communication, Blog, Contracts \& DevOps (Sprints 4-8)}

\begin{table}[h!]
\centering
\small
\renewcommand{\arraystretch}{1.3}
\begin{tabular}{|c|l|c|c|}
\hline
\rowcolor{gray!20}
\textbf{ID} & \textbf{User Story Description} & \textbf{Priority} & \textbf{Sprint} \\
\hline
US-18 & Customer views booking history & Should & 4 \\
\hline
US-19 & Customer cancels pending requests & Should & 4 \\
\hline
US-20 & Admin chats with agencies & Must & 4 \\
\hline
US-21 & Agency receives real-time messages & Must & 4 \\
\hline
US-22 & User receives in-app notifications & Should & 4 \\
\hline
US-23 & Customer contacts admin via email & Should & 4 \\
\hline
US-24 & Admin creates blog posts & Should & 5 \\
\hline
US-25 & Admin edits/deletes blogs & Should & 5 \\
\hline
US-26 & Visitor reads blogs & Should & 5 \\
\hline
US-27 & Customer comments on blogs & Could & 5 \\
\hline
US-28 & Visitor becomes Follower (email subscription for new car alerts) & Could & 5 \\
\hline
US-29 & Customer receives PDF contract & Must & 6 \\
\hline
US-30 & Contract includes QR code & Should & 6 \\
\hline
US-31 & Customer pays online & Must & 6 \\
\hline
US-32 & Agency tracks payments & Should & 6 \\
\hline
US-33 & Visitor uses chatbot & Should & 7 \\
\hline
US-34 & Customer uses multilingual chatbot & Could & 7 \\
\hline
US-35 & Admin manages agencies & Must & 7 \\
\hline
US-36 & Admin views statistics & Should & 7 \\
\hline
US-37 & Admin manages customers & Should & 7 \\
\hline
US-38 & Developer creates Docker containers & Must & 8 \\
\hline
US-39 & Developer sets up CI/CD pipelines & Must & 8 \\
\hline
US-40 & Developer deploys on Kubernetes & Should & 8 \\
\hline
\end{tabular}
\caption{Product Backlog - Part 2: Advanced Features (Sprints 4-8)}
\label{tab:backlog_part2}
\end{table}

\subsection{Backlog Summary}
\begin{table}[h!]
\centering
\begin{tabular}{|l|c|c|}
\hline
\textbf{Priority Level} & \textbf{Number of Stories} & \textbf{Total Story Points} \\
\hline
Must Have & 22 & 132 \\
\hline
Should Have & 14 & 63 \\
\hline
Could Have & 4 & 14 \\
\hline
\textbf{Total} & \textbf{40} & \textbf{209} \\
\hline
\end{tabular}
\caption{Product Backlog Summary by Priority}
\end{table}

% -------------------------------------------------------------
% SECTION: Sprint Planning Details
% -------------------------------------------------------------
\section{Sprint Planning}
The project was executed over 8 sprints, each lasting 10-15 days. The team velocity stabilized at approximately 26 story points per sprint after the initial sprints.

\subsection{Sprint Overview}
\begin{table}[h!]
\centering
\renewcommand{\arraystretch}{1.3}
\begin{tabular}{|c|l|c|c|c|}
\hline
\rowcolor{gray!20}
\textbf{Sprint} & \textbf{Goal} & \textbf{Duration} & \textbf{Planned SP} & \textbf{Completed SP} \\
\hline
1 & Authentication \& Setup & 12 days & 24 & 24 \\
\hline
2 & Car Management Module & 14 days & 28 & 28 \\
\hline
3 & Booking System Core & 15 days & 31 & 31 \\
\hline
4 & Communication Features & 12 days & 24 & 24 \\
\hline
5 & Blog \& Subscription & 10 days & 17 & 17 \\
\hline
6 & Contracts \& Payments & 15 days & 31 & 31 \\
\hline
7 & Chatbot \& Admin Panel & 12 days & 21 & 21 \\
\hline
8 & DevOps \& Deployment & 15 days & 34 & 34 \\
\hline
\rowcolor{gray!10}
\multicolumn{3}{|r|}{\textbf{Total (105 days)}} & \textbf{210} & \textbf{210} \\
\hline
\end{tabular}
\caption{Sprint Overview - Planned vs Completed Story Points}
\end{table}

% -------------------------------------------------------------
% SPRINT 1: Authentication & Setup
% -------------------------------------------------------------
\subsection{Sprint 1: Authentication \& Platform Setup}
\textbf{Sprint Goal:} Set up development environment and implement secure multi-role authentication using Keycloak.

\textbf{Duration:} 12 days

\begin{table}[h!]
\centering
\renewcommand{\arraystretch}{1.2}
\begin{tabular}{|c|p{5.5cm}|c|c|c|}
\hline
\rowcolor{gray!20}
\textbf{Task} & \textbf{Description} & \textbf{Assignee} & \textbf{Hours} & \textbf{Status} \\
\hline
T1.1 & Initialize Angular 16 project & Dev & 4h & Done \\
\hline
T1.2 & Set up Spring Boot 3.x backend & Dev & 4h & Done \\
\hline
T1.3 & Configure MySQL database & Dev & 6h & Done \\
\hline
T1.4 & Deploy Keycloak container & Dev & 4h & Done \\
\hline
T1.5 & Implement registration API & Dev & 8h & Done \\
\hline
T1.6 & Implement JWT login & Dev & 10h & Done \\
\hline
T1.7 & Create AuthGuard & Dev & 8h & Done \\
\hline
T1.8 & Password reset feature & Dev & 6h & Done \\
\hline
T1.9 & Login/register UI & Dev & 8h & Done \\
\hline
T1.10 & Unit testing & Dev & 6h & Done \\
\hline
\end{tabular}
\caption{Sprint 1 Backlog - Authentication \& Setup}
\end{table}

\textbf{Sprint 1 Retrospective:}
\begin{itemize}
    \item \textit{What went well:} Keycloak integration was smoother than expected.
    \item \textit{What could improve:} Initial environment setup took longer due to Docker configuration on Windows.
    \item \textit{Action items:} Document Docker setup steps for future reference.
\end{itemize}

% -------------------------------------------------------------
% SPRINT 2: Car Management
% -------------------------------------------------------------
\subsection{Sprint 2: Car Management Module}
\textbf{Sprint Goal:} Implement complete CRUD operations for car listings with image upload functionality.

\textbf{Duration:} 14 days

\begin{table}[h!]
\centering
\renewcommand{\arraystretch}{1.2}
\begin{tabular}{|c|p{5.5cm}|c|c|c|}
\hline
\rowcolor{gray!20}
\textbf{Task} & \textbf{Description} & \textbf{Assignee} & \textbf{Hours} & \textbf{Status} \\
\hline
T2.1 & Design Voiture entity & Dev & 4h & Done \\
\hline
T2.2 & Create car CRUD endpoints & Dev & 8h & Done \\
\hline
T2.3 & Image upload service & Dev & 6h & Done \\
\hline
T2.4 & Car listing component & Dev & 6h & Done \\
\hline
T2.5 & Car detail page & Dev & 4h & Done \\
\hline
T2.6 & Search \& filter functionality & Dev & 8h & Done \\
\hline
T2.7 & Agency dashboard & Dev & 10h & Done \\
\hline
T2.8 & Profile update feature & Dev & 6h & Done \\
\hline
T2.9 & Form validation & Dev & 4h & Done \\
\hline
T2.10 & Integration testing & Dev & 6h & Done \\
\hline
\end{tabular}
\caption{Sprint 2 Backlog - Car Management}
\end{table}

% -------------------------------------------------------------
% SPRINT 3: Booking System
% -------------------------------------------------------------
\subsection{Sprint 3: Booking System Core}
\textbf{Sprint Goal:} Implement the complete rental request workflow with availability checking and email notifications.

\textbf{Duration:} 15 days

\begin{table}[h!]
\centering
\renewcommand{\arraystretch}{1.2}
\begin{tabular}{|c|p{5.5cm}|c|c|c|}
\hline
\rowcolor{gray!20}
\textbf{Task} & \textbf{Description} & \textbf{Assignee} & \textbf{Hours} & \textbf{Status} \\
\hline
T3.1 & Design Booking entity & Dev & 4h & Done \\
\hline
T3.2 & Availability checking algorithm & Dev & 10h & Done \\
\hline
T3.3 & Booking request API & Dev & 8h & Done \\
\hline
T3.4 & Booking form with date picker & Dev & 6h & Done \\
\hline
T3.5 & Agency request management & Dev & 8h & Done \\
\hline
T3.6 & Approve/reject workflow & Dev & 6h & Done \\
\hline
T3.7 & Configure SMTP email & Dev & 4h & Done \\
\hline
T3.8 & Email notification templates & Dev & 6h & Done \\
\hline
T3.9 & Booking history page & Dev & 6h & Done \\
\hline
T3.10 & End-to-end testing & Dev & 8h & Done \\
\hline
\end{tabular}
\caption{Sprint 3 Backlog - Booking System}
\end{table}

% -------------------------------------------------------------
% SPRINT 4: Communication Features
% -------------------------------------------------------------
\subsection{Sprint 4: Communication Features}
\textbf{Sprint Goal:} Implement real-time chat system and notification infrastructure.

\textbf{Duration:} 12 days

\begin{table}[h!]
\centering
\renewcommand{\arraystretch}{1.2}
\begin{tabular}{|c|p{5.5cm}|c|c|c|}
\hline
\rowcolor{gray!20}
\textbf{Task} & \textbf{Description} & \textbf{Assignee} & \textbf{Hours} & \textbf{Status} \\
\hline
T4.1 & Design ChatMessage entity & Dev & 3h & Done \\
\hline
T4.2 & HTTP Polling configuration & Dev & 8h & Done \\
\hline
T4.3 & Chat service \& endpoints & Dev & 6h & Done \\
\hline
T4.4 & Chat UI component & Dev & 8h & Done \\
\hline
T4.5 & Notification entity \& service & Dev & 6h & Done \\
\hline
T4.6 & Notification dropdown & Dev & 4h & Done \\
\hline
T4.7 & Contact admin feature & Dev & 4h & Done \\
\hline
T4.8 & Booking cancel functionality & Dev & 4h & Done \\
\hline
T4.9 & Polish chat interface & Dev & 4h & Done \\
\hline
T4.10 & Real-time testing & Dev & 6h & Done \\
\hline
\end{tabular}
\caption{Sprint 4 Backlog - Communication Features}
\end{table}

% -------------------------------------------------------------
% SPRINT 5: Blog System
% -------------------------------------------------------------
\subsection{Sprint 5: Blog \& Subscription System}
\textbf{Sprint Goal:} Implement blog management and email subscription for visitors.

\textbf{Duration:} 10 days

\begin{table}[h!]
\centering
\renewcommand{\arraystretch}{1.2}
\begin{tabular}{|c|p{5.5cm}|c|c|c|}
\hline
\rowcolor{gray!20}
\textbf{Task} & \textbf{Description} & \textbf{Assignee} & \textbf{Hours} & \textbf{Status} \\
\hline
T5.1 & Design Blog entity & Dev & 3h & Done \\
\hline
T5.2 & Blog CRUD API endpoints & Dev & 6h & Done \\
\hline
T5.3 & Admin blog management & Dev & 6h & Done \\
\hline
T5.4 & Public blog listing & Dev & 4h & Done \\
\hline
T5.5 & Blog detail with comments & Dev & 6h & Done \\
\hline
T5.6 & Comment functionality & Dev & 4h & Done \\
\hline
T5.7 & Follower entity & Dev & 2h & Done \\
\hline
T5.8 & Subscription API & Dev & 4h & Done \\
\hline
T5.9 & Subscription form & Dev & 3h & Done \\
\hline
T5.10 & Feature testing & Dev & 4h & Done \\
\hline
\end{tabular}
\caption{Sprint 5 Backlog - Blog \& Subscription}
\end{table}

% -------------------------------------------------------------
% SPRINT 6: Contracts & Payments
% -------------------------------------------------------------
\subsection{Sprint 6: Contracts \& Payments}
\textbf{Sprint Goal:} Implement PDF contract generation with QR codes and online payment integration.

\textbf{Duration:} 15 days

\begin{table}[h!]
\centering
\renewcommand{\arraystretch}{1.2}
\begin{tabular}{|c|p{5.5cm}|c|c|c|}
\hline
\rowcolor{gray!20}
\textbf{Task} & \textbf{Description} & \textbf{Assignee} & \textbf{Hours} & \textbf{Status} \\
\hline
T6.1 & Research PDF libraries (iText) & Dev & 4h & Done \\
\hline
T6.2 & Contract PDF template & Dev & 6h & Done \\
\hline
T6.3 & PDF generation service & Dev & 10h & Done \\
\hline
T6.4 & QR code integration (ZXing) & Dev & 6h & Done \\
\hline
T6.5 & Payment microservice & Dev & 6h & Done \\
\hline
T6.6 & Payment API endpoints & Dev & 10h & Done \\
\hline
T6.7 & Payment email template & Dev & 4h & Done \\
\hline
T6.8 & Payment webhook & Dev & 6h & Done \\
\hline
T6.9 & Auto-send PDF on payment & Dev & 4h & Done \\
\hline
T6.10 & Payment-to-contract testing & Dev & 8h & Done \\
\hline
\end{tabular}
\caption{Sprint 6 Backlog - Contracts \& Payments}
\end{table}

% -------------------------------------------------------------
% SPRINT 7: Chatbot & Admin
% -------------------------------------------------------------
\subsection{Sprint 7: Chatbot \& Admin Panel}
\textbf{Sprint Goal:} Integrate AI chatbot and complete admin dashboard functionalities.

\textbf{Duration:} 12 days

\begin{table}[h!]
\centering
\renewcommand{\arraystretch}{1.2}
\begin{tabular}{|c|p{5.5cm}|c|c|c|}
\hline
\rowcolor{gray!20}
\textbf{Task} & \textbf{Description} & \textbf{Assignee} & \textbf{Hours} & \textbf{Status} \\
\hline
T7.1 & Flask chatbot project setup & Dev & 4h & Done \\
\hline
T7.2 & OpenAI ChatGPT integration & Dev & 6h & Done \\
\hline
T7.3 & Chatbot REST endpoints & Dev & 4h & Done \\
\hline
T7.4 & Chatbot UI widget & Dev & 6h & Done \\
\hline
T7.5 & Multilingual support (FR/EN) & Dev & 4h & Done \\
\hline
T7.6 & Agency management (Admin) & Dev & 6h & Done \\
\hline
T7.7 & Customer management (Admin) & Dev & 4h & Done \\
\hline
T7.8 & Statistics dashboard & Dev & 6h & Done \\
\hline
T7.9 & Data visualization charts & Dev & 4h & Done \\
\hline
T7.10 & Integration testing & Dev & 4h & Done \\
\hline
\end{tabular}
\caption{Sprint 7 Backlog - Chatbot \& Admin Panel}
\end{table}

% -------------------------------------------------------------
% SPRINT 8: DevOps & Deployment
% -------------------------------------------------------------
\subsection{Sprint 8: DevOps \& Deployment}
\textbf{Sprint Goal:} Containerize all services and establish CI/CD pipeline with Kubernetes deployment.

\textbf{Duration:} 15 days

\begin{table}[h!]
\centering
\renewcommand{\arraystretch}{1.2}
\begin{tabular}{|c|p{5.5cm}|c|c|c|}
\hline
\rowcolor{gray!20}
\textbf{Task} & \textbf{Description} & \textbf{Assignee} & \textbf{Hours} & \textbf{Status} \\
\hline
T8.1 & Dockerfile for Angular & Dev & 4h & Done \\
\hline
T8.2 & Dockerfile for Spring Boot & Dev & 4h & Done \\
\hline
T8.3 & Dockerfile for Flask Chatbot & Dev & 3h & Done \\
\hline
T8.4 & docker-compose.yml & Dev & 6h & Done \\
\hline
T8.5 & GitLab CI/CD pipeline & Dev & 10h & Done \\
\hline
T8.6 & GitLab Container Registry & Dev & 4h & Done \\
\hline
T8.7 & K8s deployment YAMLs & Dev & 10h & Done \\
\hline
T8.8 & K8s services \& ingress & Dev & 6h & Done \\
\hline
T8.9 & Minikube deployment & Dev & 8h & Done \\
\hline
T8.10 & Documentation & Dev & 4h & Done \\
\hline
\end{tabular}
\caption{Sprint 8 Backlog - DevOps \& Deployment}
\end{table}

% -------------------------------------------------------------
% SECTION: Burndown Chart Data
% -------------------------------------------------------------
\section{Sprint Burndown Analysis}
The burndown chart tracks the remaining work (in story points) over the course of each sprint. Figure \ref{fig:burndown_chart} illustrates the project-level burndown across all sprints.

\begin{figure}[h!]
    \centering
    \includegraphics[width=14cm]{img/project-burndown-chart.png}
    \caption{Project Burndown Chart - Story Points Remaining Over Sprints}
    \label{fig:burndown_chart}
\end{figure}

\subsection{Project Burndown Data}
\begin{table}[h!]
\centering
\renewcommand{\arraystretch}{1.3}
\begin{tabular}{|c|c|c|c|}
\hline
\rowcolor{gray!20}
\textbf{Sprint} & \textbf{Start SP} & \textbf{Completed SP} & \textbf{Remaining} \\
\hline
Project Start & 210 & -- & 210 \\
\hline
Sprint 1 (12 days) & 210 & 24 & 186 \\
\hline
Sprint 2 (14 days) & 186 & 28 & 158 \\
\hline
Sprint 3 (15 days) & 158 & 31 & 127 \\
\hline
Sprint 4 (12 days) & 127 & 24 & 103 \\
\hline
Sprint 5 (10 days) & 103 & 17 & 86 \\
\hline
Sprint 6 (15 days) & 86 & 31 & 55 \\
\hline
Sprint 7 (12 days) & 55 & 21 & 34 \\
\hline
Sprint 8 (15 days) & 34 & 34 & 0 \\
\hline
\rowcolor{gray!10}
\textbf{Total (105 days)} & \textbf{210} & \textbf{210} & \textbf{0} \\
\hline
\end{tabular}
\caption{Project Burndown - Story Points per Sprint}
\end{table}

\subsection{Sprint 3 Detailed Burndown (Example)}
The following table shows the daily burndown for Sprint 3 (Booking System), demonstrating how work was completed throughout the 15-day period.

\begin{table}[h!]
\centering
\renewcommand{\arraystretch}{1.2}
\begin{tabular}{|c|c|c|c|}
\hline
\rowcolor{gray!20}
\textbf{Day} & \textbf{Ideal} & \textbf{Actual} & \textbf{Done} \\
\hline
Day 1 & 28.9 & 31 & 0 \\
\hline
Day 2 & 26.8 & 29 & 2 \\
\hline
Day 3 & 24.8 & 26 & 3 \\
\hline
Day 4 & 22.7 & 24 & 2 \\
\hline
Day 5 & 20.7 & 21 & 3 \\
\hline
Day 6 & 18.6 & 18 & 3 \\
\hline
Day 7 & 16.5 & 15 & 3 \\
\hline
Day 8 & 14.5 & 12 & 3 \\
\hline
Day 9 & 12.4 & 9 & 3 \\
\hline
Day 10 & 10.3 & 7 & 2 \\
\hline
Day 11 & 8.3 & 5 & 2 \\
\hline
Day 12 & 6.2 & 3 & 2 \\
\hline
Day 13 & 4.1 & 2 & 1 \\
\hline
Day 14 & 2.1 & 1 & 1 \\
\hline
Day 15 & 0 & 0 & 1 \\
\hline
\end{tabular}
\caption{Sprint 3 Daily Burndown Chart Data}
\end{table}

% -------------------------------------------------------------
% SECTION: Team Velocity
% -------------------------------------------------------------
\section{Team Velocity}
Team velocity measures the amount of work completed per sprint, expressed in story points. This metric helps in future sprint planning and capacity estimation. Figure \ref{fig:velocity_chart} visualizes the velocity trend across all sprints.

\begin{figure}[h!]
    \centering
    \includegraphics[width=14cm]{img/velocity-chart.png}
    \caption{Team Velocity Chart - Story Points Completed per Sprint}
    \label{fig:velocity_chart}
\end{figure}

\begin{table}[h!]
\centering
\renewcommand{\arraystretch}{1.3}
\begin{tabular}{|c|c|c|c|}
\hline
\rowcolor{gray!20}
\textbf{Sprint} & \textbf{Duration} & \textbf{Velocity (SP)} & \textbf{Cumulative SP} \\
\hline
Sprint 1 & 12 days & 24 & 24 \\
\hline
Sprint 2 & 14 days & 28 & 52 \\
\hline
Sprint 3 & 15 days & 31 & 83 \\
\hline
Sprint 4 & 12 days & 24 & 107 \\
\hline
Sprint 5 & 10 days & 17 & 124 \\
\hline
Sprint 6 & 15 days & 31 & 155 \\
\hline
Sprint 7 & 12 days & 21 & 176 \\
\hline
Sprint 8 & 15 days & 34 & 210 \\
\hline
\rowcolor{gray!10}
\multicolumn{2}{|r|}{\textbf{Average Velocity}} & \textbf{26.25 SP} & \textbf{210 SP Total} \\
\hline
\end{tabular}
\caption{Team Velocity per Sprint}
\end{table}

\textbf{Velocity Analysis:}
\begin{itemize}
    \item The team maintained an average velocity of \textbf{26.25 story points per sprint}.
    \item Sprint 5 had lower velocity due to the complexity of blog features being lower than estimated.
    \item Sprint 8 had higher velocity due to familiarity with DevOps tasks and parallel work streams.
    \item Velocity remained consistent, indicating stable team performance and accurate estimation.
\end{itemize}

% -------------------------------------------------------------
% SECTION: Definition of Done
% -------------------------------------------------------------
\section{Definition of Done (DoD)}
A user story is considered "Done" when all the following criteria are met:

\begin{itemize}
    \item Code is written and follows project coding standards.
    \item Unit tests are written and pass successfully.
    \item Code is reviewed by at least one team member.
    \item Feature is integrated with the main branch.
    \item Feature is tested in the staging environment.
    \item Documentation is updated (API docs, README if needed).
    \item No critical or high-severity bugs remain.
    \item Product Owner accepts the feature during Sprint Review.
\end{itemize}

% -------------------------------------------------------------
% SECTION: Scrum Ceremonies Summary
% -------------------------------------------------------------
\section{Scrum Ceremonies}
The following Scrum ceremonies were conducted throughout the project:

\begin{table}[h!]
\centering
\renewcommand{\arraystretch}{1.4}
\begin{tabular}{|p{3.5cm}|p{3cm}|p{7cm}|}
\hline
\rowcolor{gray!20}
\textbf{Ceremony} & \textbf{Frequency} & \textbf{Purpose} \\
\hline
Sprint Planning & Start of sprint & Define sprint goal and select backlog items \\
\hline
Daily Stand-up & Daily (15 min) & Share progress, plans, and blockers \\
\hline
Sprint Review & End of sprint & Demo completed features to stakeholders \\
\hline
Retrospective & End of sprint & Reflect on process and identify improvements \\
\hline
Backlog Refinement & Weekly & Clarify, estimate, and prioritize backlog items \\
\hline
\end{tabular}
\caption{Scrum Ceremonies Overview}
\label{tab:scrum_ceremonies}
\end{table}
