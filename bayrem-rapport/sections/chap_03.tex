\chapter{Release 1: Core Platform Development}

\section{Introduction}
This chapter details the first release of the MyLoc platform, focusing on the foundational components: development environment setup, database design, authentication system, and core CRUD operations. Following the Scrum methodology, this release corresponds to \textbf{Sprints 1-3}, delivering the essential infrastructure upon which subsequent features are built.

The goals of Release 1 include:
\begin{itemize}
    \item Setting up the complete development environment and technology stack.
    \item Designing and implementing the database schema.
    \item Implementing secure authentication using Keycloak.
    \item Building the car management module for agencies.
    \item Creating the basic agency and customer management features.
\end{itemize}

\section{Sprint 1: Development Environment and Authentication}

\subsection{Development Environment Setup}
The development environment was carefully configured to support a modern microservices architecture with multiple technologies working together seamlessly.

\subsubsection{Frontend Environment}
\begin{itemize}
    \item \textbf{Angular 16}: Chosen for its robust component-based architecture and TypeScript support.
    \item \textbf{Node.js 20 LTS}: Runtime environment for Angular CLI and build tools.
    \item \textbf{npm}: Package manager for frontend dependencies.
    \item \textbf{Angular CLI}: For project scaffolding, component generation, and build automation.
\end{itemize}

\textbf{Key Angular Packages:}
\begin{itemize}
    \item \texttt{@angular/router}: For single-page application navigation.
    \item \texttt{@angular/forms}: For reactive form handling and validation.
    \item \texttt{@angular/common/http}: For REST API communication.
    \item \texttt{angular-oauth2-oidc}: For Keycloak OAuth2 integration.
    \item \texttt{ngx-toastr}: For user-friendly notifications.
\end{itemize}

\subsubsection{Backend Environment}
\begin{itemize}
    \item \textbf{Java 17 LTS}: The programming language for backend development.
    \item \textbf{Spring Boot 3.x}: Framework providing auto-configuration and rapid development.
    \item \textbf{Maven}: Build tool and dependency management.
    \item \textbf{MySQL 8}: Relational database for data persistence.
\end{itemize}

\textbf{Key Spring Dependencies:}
\begin{itemize}
    \item \texttt{spring-boot-starter-web}: For REST API development.
    \item \texttt{spring-boot-starter-data-jpa}: For database operations with Hibernate.
    \item \texttt{spring-boot-starter-security}: For security configuration.
    \item \texttt{spring-boot-starter-mail}: For email notification sending.
    \item \texttt{mysql-connector-java}: MySQL database driver.
\end{itemize}

\subsubsection{Development Tools}
\begin{table}[h!]
\centering
\renewcommand{\arraystretch}{1.3}
\begin{tabular}{|l|l|p{6cm}|}
\hline
\rowcolor{gray!20}
\textbf{Tool} & \textbf{Version} & \textbf{Purpose} \\
\hline
Visual Studio Code & Latest & Primary IDE for frontend and configuration \\
\hline
IntelliJ IDEA & Community & Backend Java development \\
\hline
XAMPP & 8.x & Local MySQL database server \\
\hline
Postman & Latest & API testing and documentation \\
\hline
Git & Latest & Version control \\
\hline
Docker Desktop & Latest & Container management \\
\hline
\end{tabular}
\caption{Development Tools and Their Purposes}
\end{table}

\subsection{Authentication System with Keycloak}
Security is paramount for the MyLoc platform. We implemented a robust authentication system using Keycloak, an open-source identity and access management solution.

\subsubsection{Why Keycloak?}
\begin{itemize}
    \item \textbf{Industry Standard}: Implements OAuth2 and OpenID Connect protocols.
    \item \textbf{Centralized Identity Management}: Single source of truth for all user identities.
    \item \textbf{Role-Based Access Control (RBAC)}: Easy management of user roles and permissions.
    \item \textbf{Built-in Features}: Login, registration, password reset, session management.
    \item \textbf{Docker Support}: Easy deployment in containerized environments.
\end{itemize}

\subsubsection{Keycloak Configuration}
The Keycloak server was configured with:
\begin{itemize}
    \item \textbf{Realm}: \texttt{myloc-realm} - The security domain for our application.
    \item \textbf{Clients}: 
    \begin{itemize}
        \item \texttt{angular-frontend}: For frontend authentication.
        \item \texttt{spring-backend}: For backend API protection.
    \end{itemize}
    \item \textbf{Roles}:
    \begin{itemize}
        \item \texttt{ADMIN}: Full platform access.
        \item \texttt{AGENCY}: Agency dashboard and car management.
        \item \texttt{CUSTOMER}: Booking and profile management.
    \end{itemize}
\end{itemize}

\subsubsection{Angular AuthGuard Implementation}
Route protection in Angular ensures that only authorized users access protected pages.

\begin{figure}[h!]
    \centering
    \includegraphics[width=14cm]{img/auth-flow-diagram.png}
    \caption{Authentication Flow with Keycloak}
    \label{fig:auth_flow}
\end{figure}

\textbf{Protected Routes:}
\begin{itemize}
    \item \texttt{/admin/*}: Requires ADMIN role.
    \item \texttt{/agency/*}: Requires AGENCY role.
    \item \texttt{/customer/*}: Requires CUSTOMER role.
    \item \texttt{/}: Public access for visitors.
\end{itemize}

\section{Sprint 2: Database Design and Implementation}

\subsection{Database Architecture}
The system uses MySQL as its relational database. The schema was designed following normalization principles to ensure data integrity and minimize redundancy.

\subsubsection{Entity-Relationship Model}
The database consists of the following main entities:

\begin{figure}[h!]
    \centering
    \includegraphics[width=16cm]{img/database-er-diagram.png}
    \caption{Entity-Relationship Diagram for MyLoc Database}
    \label{fig:er_diagram}
\end{figure}

\subsubsection{Core Tables}

\paragraph{Agency Table}
Stores information about car rental agencies registered on the platform.

\begin{verbatim}
CREATE TABLE agence (
    id BIGINT AUTO_INCREMENT PRIMARY KEY,
    agency_name VARCHAR(100) NOT NULL,
    email VARCHAR(255) NOT NULL UNIQUE,
    password VARCHAR(64) NOT NULL,
    photo LONGTEXT,
    phone_number VARCHAR(20) NOT NULL,
    city VARCHAR(100),
    description TEXT,
    created_at TIMESTAMP DEFAULT CURRENT_TIMESTAMP,
    updated_at TIMESTAMP DEFAULT CURRENT_TIMESTAMP ON UPDATE CURRENT_TIMESTAMP
);
\end{verbatim}

\textbf{Field Descriptions:}
\begin{itemize}
    \item \texttt{id}: Unique identifier (auto-incremented).
    \item \texttt{agency\_name}: Business name of the agency.
    \item \texttt{email}: Login email (unique constraint).
    \item \texttt{password}: Hashed password (bcrypt).
    \item \texttt{photo}: Base64 encoded agency logo.
    \item \texttt{phone\_number}: Contact phone number.
    \item \texttt{city}: Agency location.
\end{itemize}

\paragraph{Car (Voiture) Table}
Stores vehicle information listed by agencies.

\begin{verbatim}
CREATE TABLE voiture (
    id BIGINT AUTO_INCREMENT PRIMARY KEY,
    name VARCHAR(100) NOT NULL,
    model VARCHAR(100),
    type VARCHAR(50),
    price DOUBLE NOT NULL,
    description TEXT,
    features TEXT,
    main_image LONGTEXT,
    additional_images LONGTEXT,
    agency_id BIGINT NOT NULL,
    available BOOLEAN DEFAULT TRUE,
    created_at TIMESTAMP DEFAULT CURRENT_TIMESTAMP,
    FOREIGN KEY (agency_id) REFERENCES agence(id) ON DELETE CASCADE
);
\end{verbatim}

\textbf{Field Descriptions:}
\begin{itemize}
    \item \texttt{name}: Car name/brand.
    \item \texttt{model}: Specific model.
    \item \texttt{type}: Category (SUV, Sedan, Compact, etc.).
    \item \texttt{price}: Daily rental price.
    \item \texttt{features}: JSON or comma-separated features list.
    \item \texttt{agency\_id}: Foreign key to owning agency.
\end{itemize}

\paragraph{Booking Table}
Stores rental requests and their status.

\begin{verbatim}
CREATE TABLE booking (
    id BIGINT AUTO_INCREMENT PRIMARY KEY,
    username VARCHAR(255) NOT NULL,
    user_email VARCHAR(255) NOT NULL,
    phone VARCHAR(255),
    description VARCHAR(255),
    start_date DATE NOT NULL,
    end_date DATE NOT NULL,
    price DOUBLE,
    voiture_id BIGINT NOT NULL,
    nbr_jrs INT,
    booking_status VARCHAR(50) DEFAULT 'PENDING',
    pickup_location VARCHAR(255),
    dropoff_location VARCHAR(255),
    car_name VARCHAR(255),
    agence TEXT,
    created_at TIMESTAMP DEFAULT CURRENT_TIMESTAMP,
    FOREIGN KEY (voiture_id) REFERENCES voiture(id)
);
\end{verbatim}

\textbf{Booking Status Values:}
\begin{itemize}
    \item \texttt{PENDING}: Request submitted, awaiting agency response.
    \item \texttt{APPROVED}: Agency accepted the request.
    \item \texttt{REJECTED}: Agency declined the request.
    \item \texttt{COMPLETED}: Rental period finished.
    \item \texttt{CANCELLED}: Customer cancelled the request.
\end{itemize}

\paragraph{Customer Table}
Stores registered customer information.

\begin{verbatim}
CREATE TABLE customers (
    id BIGINT AUTO_INCREMENT PRIMARY KEY,
    username VARCHAR(100) NOT NULL,
    email VARCHAR(255) NOT NULL UNIQUE,
    password VARCHAR(64) NOT NULL,
    phone VARCHAR(20),
    address TEXT,
    created_at TIMESTAMP DEFAULT CURRENT_TIMESTAMP
);
\end{verbatim}

\paragraph{Blog Table}
Stores blog posts created by administrators.

\begin{verbatim}
CREATE TABLE blog (
    id BIGINT AUTO_INCREMENT PRIMARY KEY,
    title VARCHAR(255) NOT NULL,
    img_url VARCHAR(255),
    author VARCHAR(255),
    date VARCHAR(255),
    time VARCHAR(255),
    description VARCHAR(1024),
    quote VARCHAR(255),
    created_at TIMESTAMP DEFAULT CURRENT_TIMESTAMP
);
\end{verbatim}

\paragraph{Follower Table}
Stores email subscriptions for new car notifications.

\begin{verbatim}
CREATE TABLE followers (
    id BIGINT AUTO_INCREMENT PRIMARY KEY,
    email VARCHAR(255) NOT NULL UNIQUE,
    subscribed_at TIMESTAMP DEFAULT CURRENT_TIMESTAMP,
    active BOOLEAN DEFAULT TRUE
);
\end{verbatim}

\paragraph{Chat Message Table}
Stores real-time chat messages between admin and agencies.

\begin{verbatim}
CREATE TABLE chat_messages (
    id BIGINT AUTO_INCREMENT PRIMARY KEY,
    sender_id BIGINT NOT NULL,
    sender_type VARCHAR(20) NOT NULL,
    receiver_id BIGINT NOT NULL,
    receiver_type VARCHAR(20) NOT NULL,
    message TEXT NOT NULL,
    sent_at TIMESTAMP DEFAULT CURRENT_TIMESTAMP,
    read_status BOOLEAN DEFAULT FALSE
);
\end{verbatim}

\paragraph{Notification Table}
Stores user notifications.

\begin{verbatim}
CREATE TABLE notifications (
    id BIGINT AUTO_INCREMENT PRIMARY KEY,
    user_id BIGINT NOT NULL,
    user_type VARCHAR(20) NOT NULL,
    title VARCHAR(255) NOT NULL,
    message TEXT,
    type VARCHAR(50),
    read_status BOOLEAN DEFAULT FALSE,
    created_at TIMESTAMP DEFAULT CURRENT_TIMESTAMP
);
\end{verbatim}

\subsection{Database Relationships Summary}
\begin{table}[h!]
\centering
\renewcommand{\arraystretch}{1.3}
\begin{tabular}{|l|l|l|}
\hline
\rowcolor{gray!20}
\textbf{Relationship} & \textbf{Type} & \textbf{Description} \\
\hline
Agency $\rightarrow$ Car & One-to-Many & Agency owns multiple cars \\
\hline
Car $\rightarrow$ Booking & One-to-Many & Car can have multiple bookings \\
\hline
Customer $\rightarrow$ Booking & One-to-Many & Customer can make multiple bookings \\
\hline
Admin $\rightarrow$ Blog & One-to-Many & Admin creates multiple blogs \\
\hline
User $\rightarrow$ Comment & One-to-Many & Users can post multiple comments \\
\hline
\end{tabular}
\caption{Database Relationship Summary}
\end{table}

\section{Sprint 3: Core CRUD Operations}

\subsection{Backend REST API Structure}
The Spring Boot backend follows a layered architecture for maintainability and testability.

\subsubsection{Architecture Layers}
\begin{itemize}
    \item \textbf{Controller Layer}: Handles HTTP requests and responses.
    \item \textbf{Service Layer}: Contains business logic.
    \item \textbf{Repository Layer}: Interfaces with the database using JPA.
    \item \textbf{Entity Layer}: Defines data models mapped to database tables.
    \item \textbf{DTO Layer}: Data Transfer Objects for API communication.
\end{itemize}

\begin{figure}[h!]
    \centering
    \includegraphics[width=12cm]{img/spring-architecture.png}
    \caption{Spring Boot Layered Architecture}
    \label{fig:spring_arch}
\end{figure}

\subsubsection{REST API Endpoints}
The API follows RESTful conventions with consistent URL patterns.

\paragraph{Car Management Endpoints}
\begin{table}[h!]
\centering
\footnotesize
\renewcommand{\arraystretch}{1.3}
\begin{tabular}{|l|l|p{5cm}|}
\hline
\rowcolor{gray!20}
\textbf{Method} & \textbf{Endpoint} & \textbf{Description} \\
\hline
GET & /api/cars & Get all cars \\
\hline
GET & /api/cars/\{id\} & Get car by ID \\
\hline
GET & /api/cars/agency/\{agencyId\} & Get cars by agency \\
\hline
POST & /api/cars & Create new car \\
\hline
PUT & /api/cars/\{id\} & Update car \\
\hline
DELETE & /api/cars/\{id\} & Delete car \\
\hline
GET & /api/cars/unavailable-dates/\{id\} & Get unavailable dates \\
\hline
\end{tabular}
\caption{Car Management API Endpoints}
\end{table}

\paragraph{Agency Management Endpoints}
\begin{table}[h!]
\centering
\footnotesize
\renewcommand{\arraystretch}{1.3}
\begin{tabular}{|l|l|p{5cm}|}
\hline
\rowcolor{gray!20}
\textbf{Method} & \textbf{Endpoint} & \textbf{Description} \\
\hline
GET & /api/agencies & Get all agencies \\
\hline
GET & /api/agencies/\{id\} & Get agency by ID \\
\hline
POST & /api/agencies & Register new agency \\
\hline
PUT & /api/agencies/\{id\} & Update agency \\
\hline
DELETE & /api/agencies/\{id\} & Delete agency \\
\hline
POST & /api/agencies/login & Agency login \\
\hline
\end{tabular}
\caption{Agency Management API Endpoints}
\end{table}

\subsection{Frontend Angular Components}
The Angular frontend is organized into feature modules for maintainability.

\subsubsection{Module Structure}
\begin{itemize}
    \item \textbf{CoreModule}: Services, interceptors, guards.
    \item \textbf{SharedModule}: Reusable components, pipes, directives.
    \item \textbf{AuthModule}: Login, register, password reset.
    \item \textbf{AdminModule}: Admin dashboard and management.
    \item \textbf{AgencyModule}: Agency dashboard and car management.
    \item \textbf{CustomerModule}: Customer profile and bookings.
    \item \textbf{PublicModule}: Home, car listings, blogs.
\end{itemize}

\subsubsection{Key Components}
\begin{table}[h!]
\centering
\footnotesize
\renewcommand{\arraystretch}{1.3}
\begin{tabular}{|l|p{8cm}|}
\hline
\rowcolor{gray!20}
\textbf{Component} & \textbf{Description} \\
\hline
CarListComponent & Displays all available cars with filtering \\
\hline
CarDetailComponent & Shows detailed car information and booking form \\
\hline
AgencyDashboardComponent & Agency management interface \\
\hline
CarFormComponent & Add/Edit car form with image upload \\
\hline
AdminDashboardComponent & Platform administration interface \\
\hline
LoginComponent & User authentication form \\
\hline
\end{tabular}
\caption{Key Angular Components}
\end{table}

\section{Release 1 Deliverables}

\subsection{Completed User Stories}
By the end of Release 1, the following user stories were completed:

\begin{table}[h!]
\centering
\small
\renewcommand{\arraystretch}{1.3}
\begin{tabular}{|c|l|c|}
\hline
\rowcolor{gray!20}
\textbf{ID} & \textbf{User Story} & \textbf{Status} \\
\hline
US-01 & Visitor registers an account & Done \\
\hline
US-02 & User logs in securely & Done \\
\hline
US-03 & Admin manages user roles & Done \\
\hline
US-04 & User resets password & Done \\
\hline
US-05 & User updates profile & Done \\
\hline
US-06 & Agency adds new cars & Done \\
\hline
US-07 & Agency edits car details & Done \\
\hline
US-08 & Agency deletes cars & Done \\
\hline
US-09 & Agency uploads car photos & Done \\
\hline
US-10 & Customer browses cars & Done \\
\hline
US-11 & Customer filters cars by criteria & Done \\
\hline
US-12 & Customer views car details & Done \\
\hline
\end{tabular}
\caption{Release 1 Completed User Stories}
\end{table}

\subsection{Screenshots}

\begin{figure}[h!]
    \centering
    \includegraphics[width=14cm]{img/screenshot-login.png}
    \caption{Login Page with Keycloak Integration}
    \label{fig:screenshot_login}
\end{figure}

\begin{figure}[h!]
    \centering
    \includegraphics[width=14cm]{img/screenshot-car-list.png}
    \caption{Car Listing Page with Filtering Options}
    \label{fig:screenshot_cars}
\end{figure}

\begin{figure}[h!]
    \centering
    \includegraphics[width=14cm]{img/screenshot-agency-dashboard.png}
    \caption{Agency Dashboard for Fleet Management}
    \label{fig:screenshot_agency}
\end{figure}

\section{Conclusion}
Release 1 successfully established the foundation of the MyLoc platform. The development environment was configured with modern tools and frameworks, a robust database schema was implemented, secure authentication was integrated using Keycloak, and core CRUD operations for cars and agencies were completed.

The Scrum methodology ensured iterative delivery with regular feedback, allowing for continuous improvement. All 12 user stories planned for Sprints 1-3 were completed, achieving 100\% of the release goals.

The next chapter will cover Release 2, focusing on the booking system, email notifications, PDF contract generation, and real-time chat functionality.
